Fully homomorphic encryption schemes are usually divided into four generations. The first generation started in 2009 when Gentry \cite{cite:gentry1} proposed the first FHE scheme. Since then, a bunch of other schemes have been invented, and this chapter will present some of the more common ones.

%%%%%%%% First Generation %%%%%%%%
\section{First Generation}
The first generation of FHE schemes can be further divided into two categories:
\begin{itemize}
    \item Ideal lattice-based
    \item Approximate greatest common divisor problem ($\mathsf{AGCD}$) based
\end{itemize}

All first-generation schemes rely on the assumption that the $\mathsf{SSSP}$ problem is a hard problem. The second category also relies on another problem called the $\mathsf{AGCD}$ problem \cite{cite:QianFHE}.

First-generation schemes are quite inefficient, and today they have been replaced by more efficient FHE schemes. For example, the implementation of Gentry's first scheme in \cite{cite:gentry1}, which is ideal lattice-based, needed 30 minutes to bootstrap \cite{cite:gentry_impl}.


%\subsection{Ideal Lattice Based}
%\url{https://eprint.iacr.org/2013/250.pdf}

%\subsection{Approximate Greatest Common Divisor Problem Based}
%\cite{cite:QianFHE}
%DGHV scheme

%%%%%%%% Second Generation %%%%%%%%
\section{Second Generation}
The second-generation FHE schemes are based on the $\mathsf{LWE}$ problem and the $\mathsf{RLWE}$ problem. The development of schemes in this generation was, according to Marcolla et al. \cite{cite:QianFHE}, started by Brakerski and Vaikuntanathan, who published some initial papers \cite{cite:bv_lwe} and \cite{cite:bv_rlwe} about FHE with $\mathsf{LWE}$ and $\mathsf{RLWE}$, respectively. The subsequent research and development of these schemes finally led to the so-called BGV and B/FV schemes, which are nowadays implemented by many open-source FHE libraries, such as HElib \cite{cite:HElib} and OpenFHE \cite{cite:openFHE}.

The initial paper \cite{cite:bv_lwe} about $\mathsf{LWE}$-based FHE was extended to a new version \cite{cite:bv_lwe_2014} in 2014. This section will start by presenting some of the main ideas from that paper before the actual BGV scheme is described. Finally, some notes about the B/FV scheme will be made.



\subsection{BV Scheme}\label{sec:bv}

The $\mathsf{LWE}$-based FHE scheme presented by Brakerski and Vaikuntanathan \cite{cite:bv_lwe_2014} is called the BV scheme. We will now describe how the encryption and decryption work in the BV scheme. %, since BGV and other second-generation schemes encrypt and decrypt similarly.

To get the encryption $c$ of a message $m \in \mathbb{Z}_2$, one computes
$$c = (\mathbf{a}, b = \langle \mathbf{a}, \mathbf{s} \rangle +2e + m) \in \mathbb{Z}_q^n \times \mathbb{Z}_q,$$
where $\mathbf{a}$ is a public vector, $\mathbf{s}$ is a secret vector and $e$ is a random noise from an error distribution $\chi$.

The ciphertext $c$ can then be decrypted as 
$$m' = (b - \langle \mathbf{a}, \mathbf{s} \rangle \mod{q}) \mod{2}.$$
If the error term is small enough, we can assume that $e < q/2$, and then we see that the decryption succeeds because
\begin{align*}
m' &= (b - \langle \mathbf{a}, \mathbf{s} \rangle \mod{q}) \mod{2} \\
&= (2e + m \mod{q}) \mod{2} \\
&= 2e + m \mod 2 \\
&= m.
\end{align*}

One advantage of the BV scheme is that it, in contrast to the FHE schemes in the first generation, does not rely on the $\mathsf{SSSP}$ problem \cite{cite:QianFHE}. In fact, it only relies on the $\mathsf{LWE}$ problem, which, as we have seen, can be reduced to the shortest vector problem ($\mathsf{SVP}$) on arbitrary lattices \cite{cite:bv_lwe}.

There is also an $\mathsf{RLWE}$-based\footnote[1]{To be precise, it is based on the so-called $\mathsf{PLWE}$ problem, but this problem relies on $\mathsf{RLWE}$ problem, i.e. if $\mathsf{RLWE}$ can be solved, $\mathsf{PLWE}$ can also be solved \cite{cite:bv_rlwe}.} version of the BV scheme, presented by Brakerski and Vaikuntanathan \cite{cite:bv_rlwe}. %Then the ciphertexts and keys belong are elements in a ring $\mathcal{R}_q = \mathbb{Z}_q / \langle f(x) \rangle$ instead.



\subsection{BGV Scheme}\label{sec:bgv}

In 2014, Brakerski, Gentry, and Vaikuntanathan \cite{cite:bgv} introduced a new levelled fully homomorphic encryption scheme. It is called the BGV scheme, and the level of computations it can handle is set by the user and depends on the purpose, i.e. how deep the circuits to evaluate will be. The purpose of this structure is to avoid bootstrapping. However, there are also bootstrapping techniques for BGV, so that the scheme becomes fully homomorphic. Brakerski et al. \cite{cite:bgv} present one bootstrapping technique for the scheme. In Chapter \ref{sec:xzddf_intro} and \ref{sec:xzddf_corr}, we present another bootstrapping technique, based on a technique from Xiang et al. \cite{cite:fast_bootstrap_crypto23}, that can be used for BGV.

There are two versions of the BGV scheme -- one based on the $\mathsf{LWE}$ problem and one based on the $\mathsf{RLWE}$ problem. The $\mathsf{RLWE}$-based is more efficient and usually the one implemented in open-source FHE libraries \cite{cite:QianFHE}.


We will now present a simplified version of the original BGV scheme in Brakerski et al. \cite{cite:bgv}. The simplified scheme will consist of the algorithms $$\mathcal{E}_{BGV} = (\mathsf{BGV.Setup, BGV.KeyGen, BGV.Enc, BGV.Dec, BGV.Eval}),$$ and our notations and procedures are based on the BGV presentations in Brakerski et al. \cite{cite:bgv} and Marcolla et al. \cite{cite:QianFHE}. See Algorithm \ref{alg:bgv_setup} -- \ref{alg:bgv_eval} for the pseudocode.


$\mathsf{BGV.Setup}$ is described in Algorithm \ref{alg:bgv_setup}. $L$ is the level of the scheme, i.e. the maximum depth of an arithmetic circuit that the scheme can evaluate without bootstrapping. %$f(x)$ is a polynomial of degree $d$. It can for example be $f(x) = x^d + 1$ as in \cite{cite:QianFHE}.

The output of $\mathsf{BGV.Setup}$ is a list, or a ladder, of parameter sets -- one for each level in the arithmetic circuit. The main idea of Brakerski et al. \cite{cite:bgv} to achieve a levelled FHE scheme is to decrease the modulo $Q_j$ between each homomorphic operation. In this way, the size of the error also decreases, so that it does not escalate when performing many multiplications. For each modulo, there is a parameter set for encrypting and decrypting. Note that in our simplified version, $\mathcal{R}, N, n$ and $\chi$ are the same at all levels of the ladder, but this is not necessary.

\renewcommand{\Comment}[2][.42\linewidth]{\leavevmode\hfill\makebox[#1][l]{//~#2}}  %% https://tex.stackexchange.com/questions/180212/how-to-align-comments-in-algorithm-code
\begin{algorithm}[ht]
\caption{\;\;$\mathsf{BGV.Setup}$}\label{alg:bgv_setup}
\begin{algorithmic}
\Require
  \State $\lambda$ \Comment{security parameter}
  \State $L$ \Comment{number of levels}
\Ensure $params = \{params_j\}_{j=0}^L$ \Comment{a ladder of parameters}
  \State $N \gets N(\lambda)$ \Comment{degree of the ring}
  \State $\mathcal{R} \gets \mathbb{Z}[x]/(X^N+1)$ %\Comment{$f(x)$ has degree $d$}
  \State $n \gets n(\lambda)$ \Comment{dimension of  the ring}
  \State $\chi \gets \chi(\lambda)$ \Comment{error distribution over $\mathcal{R}$}
%  \State $B$ \Comment{Bound on length of elements from $\chi$} 
  \For{$j=L \;..\; 0$}
    \State $Q_j \gets Q(\lambda, j, L)$ \Comment{modulo at level $j$}
    \State $M_j \gets M(\lambda, j, L) = n \cdot \operatorname{polylog}(Q_j)$
    \State $params_j \gets (\mathcal{R}, n, \chi, Q_j, M_j)$ 
  \EndFor
\end{algorithmic}
\end{algorithm}

In Algorithm \ref{alg:bgv_keygen}, the procedure of generating a public and a private key is presented. The output is a list of key pairs -- one for each circuit level. Algorithm \ref{alg:bgv_enc} and \ref{alg:bgv_dec} then show how to use these keys to encrypt and decrypt.

The correctness of the decryption algorithm can be shown in the following way:
\begin{align*}
\mathsf{B}&\mathsf{GV.Dec}(params, sk, \mathbf{c}, j) \\
&= (\langle \mathbf{c}, \mathbf{s}_j \rangle \mod Q_j) \mod 2 \\
&= (\langle \mathbf{m} + \mathbf{r}^T \mathbf{A}_j, \mathbf{s}_j \rangle \mod Q_j) \mod 2 \\
&= (\langle \mathbf{m}, \mathbf{s}_j \rangle + \langle  \mathbf{r}^T \mathbf{A}_j, \mathbf{s}_j \rangle \mod Q_j) \mod 2 \\
&= (m + \mathbf{r}^T \mathbf{A}_j \mathbf{s}_j \mod Q_j) \mod 2 \\
&= (m + \mathbf{r}^T (\mathbf{b}_j \cdot 1 - \mathbf{B}_j \mathbf{t}_j ) \mod Q_j) \mod 2 \\
&= (m + 2 \mathbf{r}^T \mathbf{e}_j \mod Q_j) \mod 2 \\
&= [\mathbf{r}^T \mathbf{e}_j \text{ is small so } (m + 2 \mathbf{r}^T \mathbf{e}_j) \text{ does not wrap around } Q_j] \\
&= m + 2 \mathbf{r}^T \mathbf{e}_j \mod 2 \\
&= m  \\
\end{align*}


\renewcommand{\Comment}[2][.5\linewidth]{\leavevmode\hfill\makebox[#1][l]{//~#2}}  %% https://tex.stackexchange.com/questions/180212/how-to-align-comments-in-algorithm-code
\begin{algorithm}
\caption{\;\;$\mathsf{BGV.KeyGen}$}\label{alg:bgv_keygen}
\begin{algorithmic}[ht]
\Require
  \State $\{params_j\}_{j=0}^L$
\Ensure $(sk, pk)$ \Comment{private and public key pair}
%  \State $\mu = \theta(\log \lambda + \log L)$ \Comment{A constant}
  \For{$j=L \; .. \; 0$}
    \State $\mathbf{t}_j \xleftarrow{\text{s}} \chi^n$
    \State $\mathbf{s}_j \gets (1, \mathbf{t}_j[0], ..., \mathbf{t}_j[n-1]) \in \mathcal{R}_{Q_j}^{n+1}$
    \State $\mathbf{B}_j \xleftarrow{\text{s}} \mathcal{U}(\mathcal{R}_{Q_j}^{M_j \times n})$
    \State $\mathbf{e}_j \xleftarrow{\text{s}} \chi^{M_j}$
    \State $\mathbf{b}_j \gets \mathbf{B}_j \mathbf{t}_j + 2 \mathbf{e}_j$
    \State $\mathbf{A}_j \gets (\mathbf{b}_j \; || -\mathbf{B}_j) \in \mathcal{R}_{Q_j}^{M_j \times (n+1)}$
  \EndFor
  \State $sk \gets \{\mathbf{s}_j\}_{j=0}^L$
  \State $pk \gets \{\mathbf{A}_j\}_{j=0}^L$
\end{algorithmic}
\end{algorithm}


\renewcommand{\Comment}[2][.5\linewidth]{\leavevmode\hfill\makebox[#1][l]{//~#2}}  %% https://tex.stackexchange.com/questions/180212/how-to-align-comments-in-algorithm-code
\begin{algorithm}[ht]
\caption{\;\;$\mathsf{BGV.Enc}$}\label{alg:bgv_enc}
\begin{algorithmic}
\Require
  \State $params$
  \State $pk$
  \State $m \in \mathcal{R}_2$
\Ensure $\mathbf{c}$
  \State $\mathbf{m} \gets (m, 0, ..., 0) \in \mathcal{R}_{Q_L}^{n+1}$
  \State $\mathbf{r} \xleftarrow{\text{s}} \mathcal{U}\left(\mathcal{R}_2^{M_L}\right)$
  \State $\mathbf{c} \gets \mathbf{m} + \mathbf{r}^T \mathbf{A}_L \in \mathcal{R}_{Q_L}^{n+1} $
\end{algorithmic}
\end{algorithm}


\renewcommand{\Comment}[2][.5\linewidth]{\leavevmode\hfill\makebox[#1][l]{//~#2}}  %% https://tex.stackexchange.com/questions/180212/how-to-align-comments-in-algorithm-code
\begin{algorithm}[ht]
\caption{\;\;$\mathsf{BGV.Dec}$}\label{alg:bgv_dec}
\begin{algorithmic}
\Require
  \State $params$
  \State $sk$
  \State $\mathbf{c}$
  \State $j$  \Comment{level of $\mathbf{c}$} %\Comment{Index of $\mathbf{s}_j$ used for encrypting $\mathbf{c}$}
\Ensure $m$
  \State $m \gets (\langle \mathbf{c}, \mathbf{s}_j \rangle \mod Q_j) \mod 2$
\end{algorithmic}
\end{algorithm}

To evaluate a function $f$ on some encrypted data, Algorithm \ref{alg:bgv_eval} can be used. Without loss of generality, the algorithm assumes that $f$ is represented by an arithmetic circuit. The additions and multiplications are performed one at a time, and between each operation, the result is refreshed by moving to the next step of the parameter ladder. The refreshing consists of two steps, where a new ciphertext (still encrypting the same plaintext message) is computed in each step. First, it switches the key pair to the next key pair, and then it decreases the modulo under which, the message is encrypted.


\renewcommand{\Comment}[2][.5\linewidth]{\leavevmode\hfill\makebox[#1][l]{//~#2}}  %% https://tex.stackexchange.com/questions/180212/how-to-align-comments-in-algorithm-code
\begin{algorithm}[ht]
\caption{\;\;$\mathsf{BGV.Eval}$}\label{alg:bgv_eval}
\begin{algorithmic}
\Require
  \State $params$
  \State $pk$
  \State $f$ \Comment{circuit with add. and mult. gates}
  \State $(\mathbf{c}_0, ..., \mathbf{c}_{l-1})$
\Ensure $\mathbf{c}' = \mathsf{Enc}(f(\mathsf{Dec}(\mathbf{c}_0), ..., \mathsf{Dec}(\mathbf{c}_{l-1})))$
  \State \textbf{Function} add($pk, \mathbf{c}_0, \mathbf{c}_1$)
    \State \;\; $\mathbf{c}_2 \gets \mathbf{c}_0 + \mathbf{c}_1 \mod Q_j$
    \State \;\; \textbf{Return} $\mathsf{BGV.Eval.Refresh}(\mathbf{c}_2, Q_j, Q_{j-1})$
  \State \textbf{End Function}
  \State \textbf{Function} mult($pk, \mathbf{c}_0, \mathbf{c}_1$)
    \State \;\; $\mathbf{c}_2 \gets $ coefficient vector of $\langle \mathbf{c}_0 \otimes \mathbf{c}_1, \mathbf{x} \otimes \mathbf{x} \rangle$ \;\;\;\;\; // $\otimes$ is the tensor product
    \State \;\; \textbf{Return} $\mathsf{BGV.Eval.Refresh}(\mathbf{c}_2, Q_j, Q_{j-1})$
  \State \textbf{End Function}
  \State \textbf{Function} refresh($\mathbf{c}, Q_j, Q_{j-1}$)
    \State \;\; $\mathbf{c}_0 \gets \mathsf{SwitchKey}(\mathbf{c}, Q_j)$ \Comment{switch key to $\mathbf{s}_{j-1}$}%, still mod $Q_j$}
    \State \;\; $\mathbf{c}_1 \gets \mathsf{SwitchMod}(\mathbf{c}_0, Q_j, Q_{j-1})$ \Comment{switch mod to $Q_{j-1}$}%, still key $\mathbf{s}_{j-1}$}
  \State \textbf{End Function}
  \State Use $\operatorname{add}()$ and $\operatorname{mult}()$ to compute the circuit $f$ on  $\mathbf{c}_0, ..., \mathbf{c}_{l-1}$ and output the result to $\mathbf{c}'$.
\end{algorithmic}
\end{algorithm}


%If the level was set too low in $\mathsf{BGV.Setup}$, and the noise has become too large, a bootstrapping must be performed. However, bootstrapping can sometimes also increase the performance, for example, if the circuit to evaluate is very deep.

One final note about the BGV scheme is that Brakerski et al.~\cite{cite:bgv} also present a batching technique, that can be used when having many blocks of encrypted data that should be evaluated with the same function $f$. Batching increases the performance a lot for some types of functions $f$. One example, given by Brakerski et al.~\cite{cite:bgv}, is when deciding whether a word is present or not in a text. If batching the words to one text block, instead of having one block for each word in the text, we do not have to do a lot of $\operatorname{OR}$ operations. To batch messages, Brakerski et al.~\cite{cite:bgv} suggest replacing the plaintext space $\mathcal{R}_2$ with a ring $\mathcal{R}_p$, where $p$ is a prime. The BGV scheme then needs some other modifications as well, see the paper by Brakerski et al. \cite{cite:bgv} for further details.


\subsection{B/FV Scheme}\label{sec:bfv}
% good description in file:///C:/Users/Simon/Documents/skola/master_thesis/sources/bfv%20and%20bgv.pdf s.39/52    978-3-030-77287-1 Protecting Privacy through Homomorphic Encryption (see fast_bs_paper for how to cite)

The BGV scheme was further developed by Fan and Vercauteren to the so-called B/FV scheme \cite{cite:bfv}, and this scheme is also implemented in many open-source FHE libraries. We refer to their paper for more details about how this scheme works.


%The scheme is somewhat similar to the BGV scheme, especially in the sense that it performs a refreshment after each multiplication to keep the noise under control. We refer to \cite{cite:bfv} for more details about how this scheme works.

Kim et al. \cite{cite:bgv_vs_bfv} compare optimized versions of the BGV and the B/FV schemes. The conclusions are that the noise grows slower in B/FV and that B/FV is faster for small plaintexts, but that BGV is faster for medium and large plaintexts.

%%%%%%%% Second Generation %%%%%%%%
\section{Third Generation}
The third generation of FHE schemes started when the GSW scheme \cite{cite:gsw} was published in 2013  \cite{cite:QianFHE}. Just as the schemes in the second generation, third-generation schemes are based on the $\mathsf{LWE}$ and $\mathsf{RLWE}$ problems. However, an important difference is that the GSW scheme has a new approach for performing homomorphic operations, using a method called \textit{approximate eigenvector method} instead \cite{cite:QianFHE}.

In this section, we will first briefly present how the GSW scheme encrypts and decrypts since all schemes in the third generation more or less are built on the basics of these techniques. Then we will describe the FHEW and the TFHE schemes, which also belong to the third-generation schemes but have taken care of some of the drawbacks of the GSW scheme.

%%%%%%%% GSW scheme %%%%%%%%
\subsection{GSW Scheme}\label{sec:gsw}

A brief presentation of the GSW scheme, based on the simplified GSW scheme from Marcolla et al. \cite{cite:QianFHE}, can be found below.

Firstly, choose a random private key $\mathbf{s} = (1, s_1, ..., s_{n-1}) \in \mathbb{Z}_q^n$, and let $A \in \mathbb{Z}_q^{n \times n}$ be the public key, chosen so that $A \cdot \mathbf{s} = \mathbf{e} \approx 0$.

The encryption $C$ of a message $m \in \mathbb{Z}_q$ is then computed as $C = mI_n + RA$, where $I_n$ is the identity matrix and $R \in \mathbb{Z}_2^{n \times n}$ is randomly chosen matrix.

To decrypt, one computes $C \mathbf{s} = m I_n \mathbf{s} + RA \mathbf{s} = m I_n \mathbf{s} + R \mathbf{e} \approx m I_n \mathbf{s}$, because both $R \in \mathbb{Z}_2^{n \times n}$ and $\mathbf{e}$ are small. Since, $m I_n \mathbf{s} = (ms_0, ..., ms_{n-1}) = (m, ms_1, ..., ms_{n-1})$, one simply outputs the first element of the vector $C \mathbf{s}$ as the decryption.

The error when performing homomorphic computations grows slower in the GSW scheme than in second-generation schemes \cite{cite:QianFHE}. However, the ciphertexts are large compared to the plaintext, leading to high communication costs, and the time complexity of homomorphic operations is quite high.


%%%%%%%% FHEW scheme %%%%%%%%
\subsection{FHEW Scheme}\label{sec:fhew}

Ducas and Micciancio \cite{cite:fhew} propose some improvements to the GSW scheme, calling the new scheme FHEW. It is provided with some new techniques to achieve fast bootstrapping. One is that it provides a method for homomorphically computing $\operatorname{NAND}$ with a very low noise growth. $\operatorname{NAND}$ is functionally complete, and therefore, any function can be represented as a circuit of $\operatorname{NAND}$ gates. In FHEW, a small refresh is performed on each output of a gate. We will now briefly describe how the $\operatorname{NAND}$ operation is performed on ciphertexts.

The FHEW scheme from Ducas and Micciancio \cite{cite:fhew} encrypts with standard Regev LWE encryption (see (\ref{eq:regev_lwe}) and Definition \ref{def:lwe_encr}). Let $\mathsf{LWE}_{\mathbf{s}}^{t/q}(m,E) \subset \mathbb{Z}_q^{n+1}$ denote the set of LWE encryptions $c = (\mathbf{a}, b)$ of the message $m \in \{0,1\}$ under the private key $\mathbf{s} \in \mathbb{Z}_q^n$ such that the absolute value of the noise of ciphertext $c$ is less than $E$. Then, the homomorphic $\operatorname{NAND}$ operation described by Ducas and Micciancio~\cite{cite:fhew} is defined as:
\begin{align*}
    \operatorname{HomNAND}: \;\; &\mathsf{LWE}_{\mathbf{s}}^{4/q}(m_0, q/16) \times \mathsf{LWE}_{\mathbf{s}}^{4/q}(m_1, q/16) \to \mathsf{LWE}_{\mathbf{s}}^{2/q}(m_0 \barwedge m_1, q/4) \\
    &((\mathbf{a_0}, b_0), (\mathbf{a_1}, b_1)) \mapsto \left(-\mathbf{a_0}-\mathbf{a_1}, \frac{5q}{8} - b_0 - b_1\right),
\end{align*}

where $\barwedge$ denotes the $\operatorname{NAND}$ operator. We see that one then needs to transform the output $c \in \mathsf{LWE}_{\mathbf{s}}^{2/q}(m, q/4)$ to a ciphertext $c' \in \mathsf{LWE}_{\mathbf{s}}^{4/q}(m, q/16)$ again. Using Gentry's bootstrapping technique \cite{cite:gentry1} as usual, one can do so by homomorphically decrypting $c$ under an encryption corresponding to $\mathsf{LWE}_{\mathbf{s}}^{4/q}(m, q/16)$.
%We see that to avoid the noise from growing with the circuit depth, for each gate, one needs to transform the output $\mathbf{c} \in \mathsf{LWE}_{\mathbf{s}}^{2/q}(m, q/4)$ to a ciphertext $\mathbf{c}' \in \mathsf{LWE}_{\mathbf{s}}^{4/q}(m, q/16)$. Using Gentry's bootstrapping technique from \cite{cite:gentry1} as usual, one can do so by homomorphically decrypting $\mathbf{c}$ under an encryption corresponding to $\mathsf{LWE}_{\mathbf{s}}^{4/q}(m, q/16)$.

As noted by Ducas and Micciancio \cite{cite:fhew}, the $\operatorname{NAND}$ operation itself is very fast -- what takes time is the bootstrapping afterwards.


%%%%%%%% TFHE scheme %%%%%%%%
\subsection{TFHE Scheme}\label{sec:tfhe}

Chillotti, Gama, Georgieva, and Izabachène \cite{cite:tfhe} improve the FHEW scheme further by using the real torus $\mathbb{T}$ in different ways. They call the new technique TFHE, where 'T' stands for torus. In the paper, three versions of the TFHE scheme are proposed -- the TLWE scheme, based on a generalization of the $\mathsf{LWE}$ problem for the torus; the TRLWE scheme, which is the ring version of TLWE; and the TRGSW scheme, which is an improvement of the GSW scheme, based on rings and a torus. 

In Algorithm \ref{alg:trlwe_setup} -- \ref{alg:trlwe_eval}, we present how the TRLWE scheme works. One can simply switch between TRLWE and TLWE by just changing $\mathcal{T}$ to the real torus $\mathbb{T}$, changing $R$ to $\mathbb{Z}$ and letting $\chi$ be $\{0,1\}$-bounded instead of $B$-bounded \cite{cite:QianFHE}.

\renewcommand{\Comment}[2][.5\linewidth]{\leavevmode\hfill\makebox[#1][l]{//~#2}}  %% https://tex.stackexchange.com/questions/180212/how-to-align-comments-in-algorithm-code
\begin{algorithm}[ht]
\caption{\;\;$\mathsf{TRLWE.Setup}$}\label{alg:trlwe_setup}
\begin{algorithmic}
\Require
  \State $\lambda$ \Comment{security parameter}
\Ensure $params$ \Comment{a tuple of parameters}
  \State $k \gets k(\lambda) \in \mathbb{N}^*$
  \State $N \gets 2^k$ \Comment{degree of ring}  
  \State $\mathcal{R} \gets \mathbb{Z}[X]/(X^N+1)$
  \State $\mathcal{T} \gets \mathbb{T}[X]/(X^N+1)$ \Comment{$= \mathbb{R}[X]/(X^N+1) \mod 1$}
  \State $\mathcal{R}_2 \gets \mathbb{Z}_2[X]/(X^N+1)$ 
  \State $n \gets n(\lambda)$ \Comment{dimension of ring}
  \State $M \gets M(\lambda)$
  \State $\chi \gets \chi(\lambda)$ \Comment{$B$-bounded distribution over $\mathcal{T}$}
  \State $params \gets (\mathcal{R}, \mathcal{T}, \mathcal{R}_2, n, M, \chi)$ 
\end{algorithmic}
\end{algorithm}

\renewcommand{\Comment}[2][.5\linewidth]{\leavevmode\hfill\makebox[#1][l]{//~#2}}  %% https://tex.stackexchange.com/questions/180212/how-to-align-comments-in-algorithm-code
\begin{algorithm}[ht]
\caption{\;\;$\mathsf{TRLWE.KeyGen}$}\label{alg:trlwe_keygen}
\begin{algorithmic}
\Require
  \State $params$
\Ensure $(sk, pk)$ \Comment{private and public key pair}
  \State $\mathbf{s} \xleftarrow{\text{s}} \mathcal{U}(\mathcal{R}_2^n)$ \Comment{choose a random private key}
  \State $A \xleftarrow{\text{s}} \mathcal{U}(T^{M \times n})$
  \State $\mathbf{e} \xleftarrow{\text{s}} \chi^M$
  \State $D \gets (A \; || \; A\mathbf{s} + \mathbf{e}) \in \mathcal{T}^{M \times (n+1)}$
  \State $sk \gets \mathbf{s}$
  \State $pk \gets D$
\end{algorithmic}
\end{algorithm}

\renewcommand{\Comment}[2][.5\linewidth]{\leavevmode\hfill\makebox[#1][l]{//~#2}}  %% https://tex.stackexchange.com/questions/180212/how-to-align-comments-in-algorithm-code
\begin{algorithm}[ht]
\caption{\;\;$\mathsf{TRLWE.Enc}$}\label{alg:trlwe_enc}
\begin{algorithmic}
\Require
  \State $params$
  \State $pk$
  \State $m \in \mathcal{M} \subseteq \mathcal{T}$ \Comment{$\mathcal{M}$ is the message space}
\Ensure $\mathbf{c} \in \mathcal{T}^{n+1}$
  \State $D \gets pk$
  \State $\mathbf{r} \xleftarrow{\text{s}} \mathcal{U}\left(\mathcal{R}_2^{M}\right)$
  \State $\mathbf{m} \gets (0, ..., 0, m) \in \mathcal{T}^{n+1}$
  \State $\mathbf{c} \gets  \mathbf{r}^T D + \mathbf{m} \in \mathcal{T}^{n+1}$
\end{algorithmic}
\end{algorithm}

\renewcommand{\Comment}[2][.55\linewidth]{\leavevmode\hfill\makebox[#1][l]{//~#2}}  %% https://tex.stackexchange.com/questions/180212/how-to-align-comments-in-algorithm-code
\begin{algorithm}[ht]
\caption{\;\;$\mathsf{TRLWE.Dec}$}\label{alg:trlwe_dec}
\begin{algorithmic}
\Require
  \State $params$
  \State $sk = \mathbf{s} \in \mathcal{R}_2^n$
  \State $\mathbf{c} \in \mathcal{T}^{n+1}$
\Ensure $m$
  \State $\mathcal{T}^n \times \mathcal{T} \ni (\mathbf{a}, b) \gets \mathbf{c}$
  \State $m \gets \operatorname{round}(b - \langle \mathbf{a}, \mathbf{s} \rangle)$ \Comment{round to nearest point in $\mathcal{M} \subseteq \mathcal{T}$}
\end{algorithmic}
\end{algorithm}

\renewcommand{\Comment}[2][.51\linewidth]{\leavevmode\hfill\makebox[#1][l]{//~#2}}  %% https://tex.stackexchange.com/questions/180212/how-to-align-comments-in-algorithm-code
\begin{algorithm}[ht]
\caption{\;\;$\mathsf{TRLWE.Eval\_lincomb}$}\label{alg:trlwe_eval}
\begin{algorithmic}
\Require
  \State $params$
  \State $\mathbf{c}_0, ..., \mathbf{c}_{p-1} \in \mathcal{T}^{n+1}$
  \State $f_0, ..., f_{p-1} \in \mathcal{R}$ \Comment{coefficients for linear combination}
\Ensure $c = \operatorname{Enc}\left(\sum\limits_{i=0}^{p-1} f_i \cdot (\operatorname{Dec}(\mathbf{c}_i)\right)$
  \State $c \gets \sum\limits_{i=0}^{p-1} f_i \cdot \mathbf{c}_i$
\end{algorithmic}
\end{algorithm}


$\mathsf{TRLWE.Setup}$ in Algorithm \ref{alg:trlwe_setup} describes how to set up the parameters used in the scheme. Algorithm \ref{alg:trlwe_keygen} then shows how to generate a private and a public key for the scheme. Similar to the BGV scheme, a matrix with $M$ rows is generated as the public key, and then the encryption in Algorithm \ref{alg:trlwe_enc} chooses some random rows of this matrix to form a ciphertext.

We will now explain why the decryption in Algorithm \ref{alg:trlwe_dec} works. In the key generation (Algorithm \ref{alg:trlwe_keygen}), we compute
$$D = (A \; || \; A\mathbf{s} + \mathbf{e}),$$
and in the encryption (Algorithm \ref{alg:trlwe_enc}), we compute
$$\mathbf{c} = \mathbf{r}^T D + (0, ..., 0, m).$$
Therefore, for $(\mathbf{a}, b) = \mathbf{c}$ in the decryption (Algorithm \ref{alg:trlwe_dec}), we get
\begin{align*}
     & \mathbf{a} = \mathbf{r}^T A \\
     & b = \mathbf{r}^T (A\mathbf{s} + \mathbf{e}) + m.
\end{align*}
This means, that the decryption algorithm outputs
\begin{align*}
    & b - \langle \mathbf{a}, \mathbf{s} \rangle \\
    & = \mathbf{r}^T (A\mathbf{s} + \mathbf{e}) + m - \mathbf{r}^T A\mathbf{s} \\
    & = \mathbf{r}^T \mathbf{e} + m \\
    & \approx m,
\end{align*}
which is the plaintext message.

The evaluation function in Algorithm \ref{alg:trlwe_eval} is simple but can only handle linear combinations of messages. This is a drawback of the TRLWE scheme (and the TLWE scheme). To evaluate a non-linear function on encrypted data, the TRGSW algorithm can be used instead \cite{cite:QianFHE}.

One advantage of the TFHE scheme is that when bootstrapping, univariate functions can be evaluated at the same time. This is called programmable bootstrapping (PBS), and it means that with just one algorithm, we can both decrease the noise and evaluate a function of the ciphertext. Note that normal bootstrapping can also be seen as programmable bootstrapping with the identity function \cite{cite:tfhe_guide}, but usually, this is the only function that can be evaluated while refreshing.

Micciancio and Polyakov \cite{cite_fhew_vs_tfhe} compare the FHEW scheme with the TFHE scheme, and the conclusion they draw is that the main performance difference between the schemes mainly is due to the different bootstrapping techniques. FHEW uses AP bootstrapping, while TFHE uses GINX (these are explained more in Section \ref{sec:bootstrapping}). This results in TFHE being faster for binary and ternary messages, while FHEW is better for larger secrets. On the other hand, TFHE has a smaller bootstrapping key than FHEW \cite{cite:QianFHE}.

%%%%%%%% Second Generation %%%%%%%%
\section{Fourth Generation}
The youngest generation of the FHE schemes is the fourth one. It was started by Cheon, Kim, Kim, and Song in 2017 \cite{cite:ckks} when they published a new kind of FHE scheme, nowadays called the CKKS scheme. Since then, a lot of improvements have been suggested, but the basics of the scheme are still the same. We will now describe how it works. 

\subsection{CKKS Scheme}\label{sec:ckks}

The original CKKS scheme, from Cheon et al. \cite{cite:ckks}, is a levelled fully homomorphic encryption scheme, but a bootstrapping technique was presented later by Cheon et al. \cite{cite:ckks_bootstrap}. The difference from previous schemes is that it only computes approximate results, allowing some errors in the last decimal places when evaluating functions. We will now describe how the scheme from Cheon et al. \cite{cite:ckks} works, using the presentation of it from Marcolla et al. \cite{cite:QianFHE}.


Algorithm \ref{alg:ckks_setup} -- \ref{alg:ckks_eval} show pseudocode for the encryption scheme, consisting of $\mathcal{E}_{CKKS} = (\mathsf{CKKS.Setup, CKKS.KeyGen, CKKS.Enc, CKKS.Dec, CKKS.Eval})$.
%$(\operatorname{CKKS.Setup}, \operatorname{CKKS.KeyGen}, \operatorname{CKKS.Enc}, \operatorname{CKKS.Dec}, \operatorname{CKKS.Eval})$

$\mathcal{U}(S)$ is the uniform distribution of a set $S$, while $\mathcal{N}_{\mathbb{Z}}^N(0,\sigma^2)$ denotes a multi-dimensional discrete Gaussian distribution over $\mathbb{Z}^N$, where each component is sampled from independent discrete Gaussian distributions with variance $\sigma^2$.

$\mathcal{Z}O_{\rho}$ is also a distribution, but over $\{ -1, 0, 1 \}$, and such that $\mathbb{P}[0] = 1-\rho$ and $\mathbb{P}[1] = \mathbb{P}[-1] = \rho/2$, where $0 < \rho < 1$.

$\operatorname{HWT}(h, N)$ is a function that returns the set of signed binary vectors in $\{0, \pm 1 \}^N$ that has Hamming weight $h$, i.e. $h$ non-zero elements.

We will now explain why the decryption in Algorithm \ref{alg:ckks_dec} works. In the key generation (Algorithm \ref{alg:ckks_keygen}), we set
$$b = -as+e,$$
so when encrypting (Algorithm \ref{alg:ckks_enc}), the outputted ciphertext is
\begin{align*}
    \mathbf{c} &= (v  b+m+e_0, v  a+e_1) \\
    &= (-vas + ve + m + e_0, va + e_1).
\end{align*}
Then, since $\mathbf{s} = (1, s)$, the decryption algorithm outputs
\begin{align*}
    \langle &\mathbf{c}, \mathbf{s} \rangle \mod Q_j \\
    &= \langle (-vas + ve + m + e_0, va + e_1), (1,s) \rangle \mod Q_j \\
    &= -vas + ve + m + e_0 + vas + e_1s \mod Q_j \\
    &= m + ve + e_0 + e_1s \mod Q_j \\
    &\approx m,
\end{align*}
where the approximation at the last line can be made since the error terms are small.

\renewcommand{\Comment}[2][.5\linewidth]{\leavevmode\hfill\makebox[#1][l]{//~#2}}  %% https://tex.stackexchange.com/questions/180212/how-to-align-comments-in-algorithm-code
\begin{algorithm}[ht]
\caption{\;\;$\mathsf{CKKS.Setup}$}\label{alg:ckks_setup}
\begin{algorithmic}
\Require
  \State $\lambda$ \Comment{security parameter}
  \State $L$ \Comment{number of levels}
\Ensure $params$ \Comment{a tuple of parameters}
  \State $p \gets p(\lambda)$
  \State $Q_0 \gets Q_0(\lambda)$
  \For{$j=1 \; .. \; L$}
    \State $Q_j \gets p^jQ_0$ \Comment{$\operatorname{level}(\mathbf{c}) = j \;\; \Longrightarrow \;\; \mathbf{c} \in \mathcal{R}_{Q_j}^2$}
  \EndFor
  \State $k \gets k(\lambda, Q_L) \in \mathbb{N}^*$
  \State $N \gets 2^k$ %\Comment{Degree of ring}  
  \State $\mathcal{R} \gets \mathbb{Z}[X]/(X^N+1)$
  \State $t \gets t(\lambda, Q_L)$ \Comment{$t \in \mathbb{Z}$ is for $\mathsf{KeyGen}$}
  \State $h \gets h(\lambda, Q_L)$ \Comment{Hamming weight for private key}
%  \State $\mathcal{S}$
  \State $\sigma \gets \sigma(\lambda, Q_L)$ \Comment{variance for sampled errors}
%  \State $\mathcal{D}G(d,\sigma^2) \gets ((d,\sigma^2) \mapsto \mathcal{D}G_{d,\sigma^2})$ \Comment{Discrete Gaussians over $\mathbb{Z}^d$}
%  \State $\mathcal{Z}O(\rho) \gets (\rho \mapsto \mathcal{Z}O_{\rho})$ \Comment{Distributions over $\{ -1, 0, 1 \}$}
  %\State $\rho \gets 1/2$ %\Comment{$0 < \rho < 1$}
%  \State $\chi \gets \chi(\lambda)$ \Comment{$B$-bounded distribution}
%  \State $params \gets (d, \mathcal{R}, t, \{Q_j\}_{j=0}^L, \sigma, \mathcal{D}G(\cdot, \cdot), \mathcal{Z}O(\cdot), \chi)$
  \State $params \gets (\{Q_j\}_{j=0}^L, \mathcal{R}, N, t, h, \sigma)$
\end{algorithmic}
\end{algorithm}


\renewcommand{\Comment}[2][.5\linewidth]{\leavevmode\hfill\makebox[#1][l]{//~#2}}  %% https://tex.stackexchange.com/questions/180212/how-to-align-comments-in-algorithm-code
\begin{algorithm}[ht]
\caption{\;\;$\mathsf{CKKS.KeyGen}$}\label{alg:ckks_keygen}
\begin{algorithmic}
\Require
  \State $params$
\Ensure $(sk, pk, evk)$ \Comment{private, public, and eval. key}
%  \State $s \xleftarrow{\text{s}} \chi$
  \State $\mathcal{R} \ni s \xleftarrow{\text{s}} \mathcal{U}(\operatorname{HWT}(h, N))$ \Comment{random vector is coefficients} %\Comment{$s \in \mathcal{R}$ ($\operatorname{HWT}$ gives coefficients)}
  \State $a \xleftarrow{\text{s}} \mathcal{U}(\mathcal{R}_{Q_L})$
  \State $\mathcal{R} \ni e \xleftarrow{\text{s}} \mathcal{N}_{\mathbb{Z}}^N(0,\sigma^2)$ \Comment{random vector is coefficients}
  \State $b \gets -as+e \mod Q_L$
  \State $a' \xleftarrow{\text{s}} \mathcal{U}(\mathcal{R}_{t \cdot Q_L})$
  \State $ \mathcal{R} \ni e' \xleftarrow{\text{s}} \mathcal{N}_{\mathbb{Z}}^N(0,\sigma^2)$ \Comment{random vector is coefficients}
  \State $b' \gets -a's+e'+ts^2 \mod (t \cdot Q_L)$  
  \State $sk \gets (1,s)$ \Comment{$sk \in \mathcal{R}^2$}
  \State $pk \gets (b,a)$ \Comment{$pk \in \mathcal{R}_{Q_L}^2$}
  \State $evk \gets (b', a')$ \Comment{$evk \in \mathcal{R}_{t \cdot Q_L}^2$}
\end{algorithmic}
\end{algorithm}


\renewcommand{\Comment}[2][.5\linewidth]{\leavevmode\hfill\makebox[#1][l]{//~#2}}  %% https://tex.stackexchange.com/questions/180212/how-to-align-comments-in-algorithm-code
\begin{algorithm}[ht]
\caption{\;\;$\mathsf{CKKS.Enc}$}\label{alg:ckks_enc}
\begin{algorithmic}
\Require
  \State $params$
  \State $pk$
  \State $m \in \mathcal{R}$
\Ensure $\mathbf{c} \in \mathcal{R}_{Q_L}^2$
  \State $v \xleftarrow{\text{s}} \mathcal{Z}O(1/2)$
  \State $\mathcal{R} \ni e_0, e_1 \xleftarrow{\text{s}} \mathcal{N}_{\mathbb{Z}}^N(0,\sigma^2)$ \Comment{random vector is coefficients}
  \State $(b,a) \gets pk \in \mathcal{R}_{Q_L}^2$
  \State $\mathbf{c} \gets (v  b+m+e_0, v  a+e_1)$
\end{algorithmic}
\end{algorithm}



\renewcommand{\Comment}[2][.55\linewidth]{\leavevmode\hfill\makebox[#1][l]{//~#2}}  %% https://tex.stackexchange.com/questions/180212/how-to-align-comments-in-algorithm-code
\begin{algorithm}[ht]
\caption{\;\;$\mathsf{CKKS.Dec}$}\label{alg:ckks_dec}
\begin{algorithmic}
\Require
  \State $params$
  \State $sk = \mathbf{s} \in \mathcal{R}^2$
  \State $j$  \Comment{level of $\mathbf{c}$}
  \State $\mathbf{c} \in \mathcal{R}_{Q_j}^2$
\Ensure $m$
  %\State $(a, b) \gets \mathbf{c}$
  \State $m \gets \langle \mathbf{c}, \mathbf{s} \rangle \mod Q_j$ \Comment{$m \in \mathcal{R}$}
\end{algorithmic}
\end{algorithm}


\renewcommand{\Comment}[2][.40\linewidth]{\leavevmode\hfill\makebox[#1][l]{//~#2}}  %% https://tex.stackexchange.com/questions/180212/how-to-align-comments-in-algorithm-code
\begin{algorithm}[ht]
\caption{\;\;$\mathsf{CKKS.Eval}$}\label{alg:ckks_eval}
\begin{algorithmic}
\Require
  \State $params$
  \State $evk$
  \State $\mathbf{c}_0, ..., \mathbf{c}_{p-1}$
  \State $f$ \Comment{function to evaluate}
\Ensure $\mathbf{c}' = \mathsf{Enc}(f(\mathsf{Dec}(\mathbf{c}_0), ..., \mathsf{Dec}(\mathbf{c}_{p-1})))$
  \State
  \State \textbf{Function} rescale($\mathbf{c}, l, l'$)
    \State \;\; $\mathbf{c}' \gets \left \lfloor \frac{Q_{l'}}{Q_l}\mathbf{c} \right \rceil \mod Q_{l'}$
    \State \;\; $\textbf{Return } \mathbf{c}'$
  \State \textbf{End Function}
  \State
  \State \textbf{Function} same\_level($\mathbf{c}_0, \mathbf{c}_1$)
    \State \;\; $l_0 \gets \operatorname{level}(\mathbf{c}_0)$
    \State \;\; $l_1 \gets \operatorname{level}(\mathbf{c}_1)$
    \State \;\; \textbf{if} $l_0 < \ l_1$ \textbf{then}
    \State \;\;\;\;\; $\mathbf{c}_0 \gets \operatorname{rescale}(\mathbf{c}_0, l_0, l_1)$
    \State \;\;\;\;\; $l_0 \gets l_1$
    \State \;\; \textbf{else if} $l_0 > l_1$ \textbf{then}
    \State \;\;\;\;\; $\mathbf{c}_1 \gets \operatorname{rescale}(\mathbf{c}_1, l_1, l_0)$
    \State \;\; \textbf{endif}
    \State \;\; $\textbf{Return } (\mathbf{c}_0, \mathbf{c}_1, l_0)$
  \State \textbf{End Function}
  \State
  \State \textbf{Function} add($\mathbf{c}_0, \mathbf{c}_1, params$)
    \State \;\; $(\mathbf{c}_0, \mathbf{c}_1, l) \gets \operatorname{same\_level}(\mathbf{c}_0, \mathbf{c}_1)$ \Comment{ $ \implies \operatorname{level}(\mathbf{c}_0) = \operatorname{level}(\mathbf{c}_1)$}
    \State \;\;  $\mathbf{c}_{\text{add}} \gets \mathbf{c}_0 + \mathbf{c}_1 \mod Q_l$
    \State \;\; $\textbf{Return } \mathbf{c}_{\text{add}}$
  \State \textbf{End Function}
  \State
  \State \textbf{Function} mult($\mathbf{c}_0, \mathbf{c}_1, params, evk$)
    \State \;\; $(\mathbf{c}_0, \mathbf{c}_1, l) \gets \operatorname{same\_level}(\mathbf{c}_0, \mathbf{c}_1)$
    \State \;\; $(b_0, a_0) \gets \mathbf{c}_0$
    \State \;\; $(b_1, a_1) \gets \mathbf{c}_1$
    \State \;\; $(d_0, d_1, d_2) \gets (b_0b_1, a_0b_1+a_1b_0, a_0a_1)) \mod Q_l$
    \State \;\; $\mathbf{c}_{\text{mult}} = (d_0, d_1) + \lfloor t^{-1} \cdot d_1 \cdot evk \rceil \mod Q_l$
    \State \;\; $\textbf{Return } \mathbf{c}_{\text{mult}}$
  \State \textbf{End Function}
  \State
  \State Use $\operatorname{add}()$ and $\operatorname{mult}()$ to compute the circuit $f$ on  $\mathbf{c}_0, ..., \mathbf{c}_{p-1}$ and output the result to $\mathbf{c}'$.
\end{algorithmic}
\end{algorithm}

One last note about CKKS is that there are some attacks on it, for example from Li and Micciancio \cite{cite:ckks_attack}. As Marcolla et al. \cite{cite:QianFHE} mention, it is possible to extract the private key by just knowing a ciphertext and its corresponding plaintext. Since the error is a linear combination of the components of the key, one can compute the key with just some basic linear algebra.


\section{Comparison}
Marcolla et al. \cite{cite:QianFHE} conclude that BGV and B/FV are good choices if working with finite fields and exact modular arithmetic. However, if bootstrapping will be needed, or if non-linear functions need to be evaluated, third and fourth-generation schemes are better instead. TFHE and other third-generation schemes are usually suitable when performing bit-wise operations or evaluating Boolean circuits, while CKKS is a good choice when doing real number arithmetic. At last, second and fourth-generation schemes are usually good when doing vector or matrix computations, since these schemes are provided with packing techniques. %However, second and third-generation schemes are supposed to be best when levelled homomorphic encryption is enough since their bootstrapping is quite inefficient. ... but faster with XZDDF?

%\cite{cite:QianFHE}: "The second generation schemes, BGV and B/FV, are suitable to work with finite fields in the modular exact arithmetic. They are equipped with efficient packing which enables the use of SIMD (namely, single instruction multiple data) instructions to perform computations over vectors of integers (i.e. batching). Thus, these schemes are excellent candidates when large arrays of numbers are to be processed simultaneously. Second-generation schemes are not good candidates for circuits where bootstrapping is required (i.e. circuits with large multiplicative depth), or where non-linear functions are to be implemented. Third-generation schemes should be adopted instead, namely TFHE, which can outperform previous schemes for bit-wise operations, i.e. when computations are expressed as boolean circuits [58]. The main limitation of TFHE is the lack of support for CRT packing (i.e. batching), hence the scheme can be outperformed by previous approaches when processing large amounts of data simultaneously. The fourth generation, i.e. CKKS, is the best option for real numbers arithmetic. "

%See table 1 in \cite{qian}!!!

%\cite{cite:QianFHE} (end): "Second and fourth generation schemes are equipped with packing techniques, which make them efficient for matrix multiplication, while third generation schemes are the only ones to enable efficient evaluation of non-linear functions. Moreover, second and fourth-generation schemes are not equipped with fast bootstrapping techniques; this limits their application to their leveled version, not their fully homomorphic version. Another limitation is the absence of thorough efforts related to noise analysis, mainly for second-generation schemes. "

