
%\section{Cryptographic Notations}


\section{Cryptographic Notations and Definitions}

In this section, some notations and definitions related to fully homomorphic encryption are introduced. %In addition, an overview of the properties of fully homomorphic encryption schemes will be given.

First of all, let $\mathcal{E}$ denote an encryption scheme that can encrypt and decrypt messages and ciphertexts, respectively. The set of messages (plaintexts) that the cryptosystem can encrypt is denoted by $\mathcal{M}$. Similarly, denote the set of ciphertexts that the system can decrypt by $\mathcal{C}$.

Also, for a given encryption scheme $\mathcal{E}$, let the functions
\begin{align*} 
&\operatorname{Enc}: \; \mathcal{M} \to \mathcal{C} \\
&\operatorname{Dec}: \; \mathcal{C} \to \mathcal{M}
\end{align*}
denote the encryption function and the decryption function, respectively. This means that $\operatorname{Dec}(\operatorname{Enc}(x)) = x$ for any message $x \in \mathcal{M}$.

In the case of symmetric-key cryptography, one can use the notations $\operatorname{Enc}_k$ and $\operatorname{Dec}_k$ to emphasize the need for the secret key $k$ to encrypt and decrypt. Similarly, we write $\operatorname{Enc}_{\mathsf{pk}}$ and $\operatorname{Dec}_{\mathsf{sk}}$ in public-key cryptography, where $\mathsf{sk}$ is the private key and $\mathsf{pk}$ is the public key.

Now, fully homomorphic encryption can be defined.
\begin{definition}[Fully homomorphic encryption]\label{def:fhe}
An encryption scheme $\mathcal{E}$ is called a fully homomorphic encryption (FHE) scheme if its encryption function $\operatorname{Enc}: \; \mathcal{M} \to \mathcal{C}$ preserves the two operations addition and multiplication, i.e. 
\begin{align*} 
&\operatorname{Enc}(x+y) = \operatorname{Enc}(x) \oplus \operatorname{Enc}(y) \\
&\operatorname{Enc}(x \cdot y) = \operatorname{Enc}(x) \odot \operatorname{Enc}(y),
\end{align*}
for some operations $\oplus$ and $\odot$ on the set of ciphertexts.
\end{definition}

This means that in fully homomorphic encryption schemes, computations with encrypted messages can be performed without the need of first decrypting them.

Being able to perform both addition and multiplication means that we can also compute $\operatorname{XOR}$ and $\operatorname{AND}$ operations. Since these two operators form a functionally complete set of Boolean operators, i.e. they can express all possible truth tables, any Boolean circuit can be computed in a fully homomorphic encryption scheme \cite{cite:QianFHE}.

Now, the following notations are introduced
\begin{align*} 
& \mathscr{M} = (\mathcal{M}, +, \times) \\
& \mathscr{C} = (\mathcal{C}, \oplus, \otimes).
\end{align*}
$\mathscr{M}$ is a mathematical structure consisting of the plaintext set $\mathcal{M}$ associated with the addition and multiplication operation. Similarly, $\mathscr{C}$ is the cipher text space, consisting of the corresponding two operators $\oplus$ and $\otimes$, operating on ciphertexts in $\mathcal{C}$. In practical implementations, we usually have that $\oplus = +$ and $\otimes = \times $.

We also define some weaker forms of homomorphic encryption.
\begin{definition}[Somewhat homomorphic encryption]\label{def:she}
An encryption scheme $\mathcal{E}$ is called a somewhat homomorphic encryption (SHE) scheme if it is homomorphic only for a limited class of circuits.
\end{definition}
%We also introduce the term \textit{levelled homomorphic encryption} (LHE), which, in contrast to SHE schemes, can evaluate any circuits, but just up to a certain depth.
\begin{definition}[Levelled homomorphic encryption]\label{def:lhe}
An encryption scheme $\mathcal{E}$ is called a levelled homomorphic encryption (LHE) scheme if it can evaluate any circuits of bounded depth, i.e. a depth lower than a predetermined value.
\end{definition}
% better def levelled fully homomorphic encryption? \url{https://eprint.iacr.org/2011/344.pdf} p.13

Note that the sets of SHE schemes and LHE schemes are not disjoint.


At last, the security parameter is defined.
\begin{definition}[Security parameter]\label{def:security_parameter}
The security parameter $\lambda$ in an encryption scheme $\mathcal{E}$ is a parameter in bits that decides the sizes of the other parameters in $\mathcal{E}$ so that the time complexity for breaking the system becomes $\mathcal{O}(2^{\lambda})$.
\end{definition}
%Usually, $\lambda$ is in bits, and if, for example, $\lambda = 128$ bits, it means that the time complexity for breaking the system is $\mathcal{O}(2^{128})$


%%%%%%%% Structure of FHE %%%%%%%%
\section{Structure of Fully Homomorphic Encryption Schemes}

Usually, FHE schemes are based on the $\mathsf{LWE}$ problem. This means that each ciphertext has some noise associated with it, and the noise will increase each time an operation is performed. If too many operations are computed, the cumulative noise will become so big that the decryption fails. Luckily, there are solutions to avoid this by decreasing the noise before it becomes too large. We say that a ciphertext is \textit{refreshed} when we decrease its noise.

Without refreshing, the encryption scheme is homomorphic only for a limited number of operations, i.e. it is just an SHE scheme.

Gentry \cite{cite:gentry1} shows that any SHE scheme can be modified to become fully homomorphic. To refresh the ciphertexts, he introduces a technique called \textit{bootstrapping}.

The general FHE scheme Gentry \cite{cite:gentry1} suggests consists of the four algorithms
$$\mathcal{E} = (\operatorname{KeyGen}, \operatorname{Enc}, \operatorname{Dec}, \operatorname{Eval}),$$
where all components are probabilistic polynomial-time (PPT) algorithms \cite{cite:QianFHE}. As before, $\operatorname{Enc}$ and $\operatorname{Dec}$ denote the encryption and the decryption functions, respectively. %However, the range of the $\operatorname{Dec}$ function is extended to $\mathcal{M} \cup \{\perp\}$, where $\perp$ means decryption failure (eg. because of too much noise).

$\operatorname{KeyGen}$ is a function that takes a security parameter $\lambda$ as input and outputs
$$\operatorname{KeyGen}: \; \lambda \mapsto (\mathsf{sk}, \mathsf{pk}, \mathsf{evk}),$$
where $\mathsf{sk}$ is a private key, $\mathsf{pk}$ is a public key, and $\mathsf{evk}$ is a public evaluation key. The latter is used to evaluate functions on ciphertexts homomorphically.

$\operatorname{Eval}$ is a function that takes the following input:
\begin{itemize}
    \item An evaluation key $\mathsf{evk}$
    \item A function $f: \mathcal{M}^t \to \mathcal{M}$ taking $t$ inputs $x_1, ..., x_t \in \mathcal{M}$
    \item $t$ ciphertexts $(c_1, ..., c_t).$
\end{itemize}
It then, with overwhelming probability, outputs
$$\operatorname{Eval}_{\mathsf{evk}}: \; (f, (c_1, ..., c_t)) \mapsto \operatorname{Enc}_{\mathsf{pk}}\bigl(f(\operatorname{Dec}_{\mathsf{sk}}(c_1), ..., \operatorname{Dec}_{\mathsf{sk}}(c_t))\bigr).$$
This means that $\operatorname{Eval}$ is a function that takes a given number of ciphertexts as input and outputs the encrypted value that $f$ would output if inputting the corresponding plaintexts to it. Note that $\operatorname{Eval}$ does this without any knowledge about the private key $\mathsf{sk}$.

Appendix \ref{sec:fhe_schemes} contains theory about some of the common FHE schemes.

%[TODO??] funtions in $\mathcal{F}$, maximum cipher bit-length...? bottom p.6 in \cite{cite:QianFHE}


\section{Introduction to Bootstrapping}

%When having performed many homomorphic operations on a ciphertext in a somewhat homomorphic encryption scheme, the noise has increased.
The bootstrapping technique that Gentry \cite{cite:gentry1} proposes for reducing the noise is illustrated in Figure \ref{fig:bs_tikz} below. Algorithm \ref{alg:bootstrap} shows more specifically how the method works. Note that the algorithm does not contain any secret data, and hence, it may be performed by a third party.


\begin{figure}[ht]
    \centering\resizebox{\textwidth}{!}{%
        \begin{tikzpicture}
    
    % Define and set local variables
    \newlength{\lineWidth} \setlength{\lineWidth}{1pt}
    \newlength{\ctWidth} \setlength{\ctWidth}{13mm}
    \newlength{\ctHeight} \setlength{\ctHeight}{36mm}
    \newlength{\msgWidth} \setlength{\msgWidth}{\ctWidth}
    \newlength{\msgHeight} \setlength{\msgHeight}{7mm}
    \newlength{\noiseWidth} \setlength{\noiseWidth}{\ctWidth}%\addtolength{\noiseWidth}{-1.13pt}
    \newlength{\noiseHeightHalf} \setlength{\noiseHeightHalf}{4mm}
    \newlength{\noiseHeightFilled} \setlength{\noiseHeightFilled}{\ctHeight}\addtolength{\noiseHeightFilled}{-\msgHeight}
    \newlength{\ctDist} \setlength{\ctDist}{1.8\ctWidth}
    
    \tikzset{
        outer/.style={draw = gray, dashed, fill = green!2, thick, inner sep = 10pt, rounded corners = 10pt, minimum height = 0mm},
        ct/.style = {draw, solid, line width = \lineWidth, inner sep = 3pt, minimum width = \ctWidth, minimum height = \ctHeight, rounded corners = 0pt},
        msg/.style = {draw, solid, fill = white!100, line width = \lineWidth, inner sep = 3pt, minimum width = \msgWidth, minimum height = \msgHeight, rounded corners = 0pt},
        noise/.style = {draw = none, solid, thick, inner sep=3pt, minimum width = \noiseWidth, rounded corners=0pt},
        arrow/.style={->, >=stealth, thick}
    }
    
    % First-layer
    \node[outer] (firstlayer) {
        \begin{tikzpicture}[node distance=0.5cm, outer sep = 0pt]

            % Ciphertext 11
            \node[msg] (msg11node) at (0,0) {$m$};
            \node[noise, fill = green!75!gray, anchor=south, minimum height = \noiseHeightHalf] (noise11node) at (msg11node.north) {noise};
            \node[ct, anchor=south] (ct11node) at (msg11node.south) {};

            % Ciphertext 12
            \node[msg, anchor=south] (msg12node) at ([xshift=\ctDist] ct11node.south east) {$f(m)$};
            \node[noise, fill = green!75!gray, anchor=south, minimum height = \noiseHeightFilled] (noise12node) at (msg12node.north) {noise};
            \node[ct, anchor=south] (ct12node) at (msg12node.south) {};

            % Arrow between ct11 and ct12
            \draw[arrow, solid] (ct11node) -- node[midway, yshift=3mm](arrowNode1) {$f(\cdot)$} (ct12node);

            % Text outside ciphertexts
            \node (text) [anchor=north] at ([yshift=5.5em]arrowNode1.north) {First-layer};
        \end{tikzpicture}
    };

    % Second-layer
    \node[outer, anchor=north] (secondlayer) at ([xshift=48mm] firstlayer.north east) {
        \begin{tikzpicture}[node distance=0.5cm, outer sep = 0pt]
            
            % Ciphertext 21
            \node[msg] (msg211node) at (0,0) {$m$};
            \node[noise, fill = green!50!gray!70, anchor=south, minimum height = \noiseHeightHalf] (noise211node) at (msg211node.north) {noise};
            \node[ct, anchor=south] (ct211node) at (msg211node.south) {};
            
            \node[ct, anchor=north] (ct212node) at (msg211node.north) {};
            \node[msg, anchor=south] (msg212node) at (ct212node.south) {$f(m)$};
            \node[noise, fill = green!75!gray, anchor=south, minimum height = \noiseHeightFilled] (noise212node) at (msg212node.north) {noise};
            \node[ct, anchor=north] (ct212node) at (msg211node.north) {};

            % Ciphertext 22
            \node[msg, anchor=south] (msg22node) at ([xshift=\ctDist] ct211node.south east) {$f(m)$};
            \node[noise, fill = green!50!gray!70, anchor=south, minimum height = \noiseHeightHalf] (noise22node) at (msg22node.north) {noise};
            \node[ct, anchor=south] (ct22node) at (msg22node.south) {};

            % Arrow between ct21 and ct22
            \draw[arrow, solid] (ct211node) -- node[midway, xshift=0mm, yshift=8mm](arrowNode2) {$\operatorname{Dec}(\cdot)$} (ct22node);

            % Text outside ciphertexts
            \node (text) [anchor=north] at ([yshift=21.8mm]arrowNode2.north) {Second-layer};        
        \end{tikzpicture}
    };

    % Arrow between the images
    \draw[arrow] (2.15,-0.57) -- node[midway, above] {reencryption} (5.2,-0.57);
    % \draw[arrow] (msg12node) -- node[above] {reencryption} (secondlayer);
\end{tikzpicture}
    }
    \caption{Illustration showing how bootstrapping works.}
    \label{fig:bs_tikz}
\end{figure}

\renewcommand{\Comment}[2][.5\linewidth]{\leavevmode\hfill\makebox[#1][l]{//~#2}}  %% https://tex.stackexchange.com/questions/180212/how-to-align-comments-in-algorithm-code
\begin{algorithm}
\caption{\;\;Naive bootstrapping}\label{alg:bootstrap}
\begin{algorithmic}
\Require
  \State $\overline{\mathsf{sk}} = \operatorname{Enc}_{\mathsf{pk}}(\mathsf{sk})$ \Comment{encryption of the secret key}
  \State $\mathsf{pk}$
  \State $\mathsf{evk}$
  \State $c = \operatorname{Enc}_{\mathsf{pk}}(m)$ \Comment{encrypted message to refresh}
\Ensure $c' = \operatorname{Enc}_{\mathsf{pk}}(m)$ \Comment{$c'$ has smaller noise than $c$}
\State $\overline{c} \gets \operatorname{Enc}_{\mathsf{pk}}(c)$ %\Comment{Encrypted $c$ still contain much noise}
\State $c' \gets \operatorname{Eval}_{\mathsf{evk}}(\operatorname{Dec}, \overline{c}, \overline{\mathsf{sk}}) $
%\Comment{$\operatorname{Dec}$ uses last argument as key}
\end{algorithmic}
\end{algorithm}



%\renewcommand{\Comment}[2][.5\linewidth]{\leavevmode\hfill\makebox[#1][l]{//~#2}}  %% https://tex.stackexchange.com/questions/180212/how-to-align-comments-in-algorithm-code
%\begin{algorithm}
%\caption{General Bootstrapping}\label{alg:bootstrap}
%\begin{algorithmic}
%\Require
%  \State $\overline{\mathsf{sk}_1} = \operatorname{Enc}_{\mathsf{pk}_2}(\mathsf{sk}_1)$ \Comment{Encryption of secret key 1}
%  \State $\mathsf{pk}_1$ %\Comment{The public key in key pair 1}
%  \State $\mathsf{pk}_2$ %\Comment{The public key in key pair 2}
%  \State $\mathsf{evk}_2$ %\Comment{The evaluation key to key pair 2}
  %\State $\mathsf{evk}_2$
%  \State $\operatorname{Enc}$ %\Comment{Encryption algorithm}
%  \State $\operatorname{Dec}$
%  \State $\operatorname{Eval}$ %\Comment{Evaluation algorithm for key pair 2}
%  \State $c = \operatorname{Enc}_{\mathsf{pk}_1}(m)$ \Comment{Encrypted message to refresh}
%\Ensure $c' = \operatorname{Enc}_{\mathsf{pk}_2}(m)$ \Comment{$c'$ has smaller noise than $c$}
%\State $\overline{c} \gets \operatorname{Enc}_{\mathsf{pk}_2}(c)$ %\Comment{Encrypted $c$ still contain much noise}
%\State $c' \gets \operatorname{Eval}_{\mathsf{evk}}(\operatorname{Dec}, \overline{c}, \overline{\mathsf{sk}_1}) $
%\Comment{$\operatorname{Dec}$ uses last argument as key}
%\end{algorithmic}
%\end{algorithm}

%\renewcommand{\Comment}[2][.5\linewidth]{\leavevmode\hfill\makebox[#1][l]{//~#2}}  %% https://tex.stackexchange.com/questions/180212/how-to-align-comments-in-algorithm-code
%\begin{algorithm}
%\caption{General Bootstrapping}\label{alg:bootstrap}
%\begin{algorithmic}
%\Require
%  \State $(\mathsf{sk}_1, \mathsf{pk}_1)$ \Comment{Key pair 1}
%  \State $(\mathsf{sk}_2, \mathsf{pk}_2, \mathsf{evk}_2)$ \Comment{Key pair 2 and evaluation key 2}
%  %\State $\mathsf{evk}_2$
%  \State $\operatorname{Enc}$ %\Comment{Encryption algorithm}
%  \State $\operatorname{Dec}$
%  \State $\operatorname{Eval}$ %\Comment{Evaluation algorithm for key pair 2}
%  \State $c = \operatorname{Enc}_{\mathsf{pk}_1}(m)$ \Comment{Encrypted message to refresh}
%\Ensure $c'$ \Comment{$c' := \operatorname{Enc}_{\mathsf{pk}_2}(m)$ with small noise}
%\State $\overline{\mathsf{sk}_1} \gets \operatorname{Enc}_{\mathsf{pk}_2}(\mathsf{sk}_1)$
%\State $\overline{c} \gets \operatorname{Enc}_{\mathsf{pk}_2}(c)$ %\Comment{Encrypted $c$ still contain much noise}
%\State $c' = \operatorname{Eval}_{\mathsf{evk}}(\operatorname{Dec}, \overline{c}, \overline{\mathsf{sk}_1}) $
%\Comment{$\operatorname{Dec}$ uses last argument as key}
%\end{algorithmic}
%\end{algorithm}

At the last line of Algorithm \ref{alg:bootstrap}, the new ciphertext $c'$ encrypts the same message as the inputted ciphertext $c$. This can be shown by expanding the expression assigned to $c'$:
\begin{align*}
    c' &= \operatorname{Eval}_{\mathsf{evk}}(\operatorname{Dec}, \overline{c}, \overline{\mathsf{sk}}) \\
    &= \operatorname{Eval}_{\mathsf{evk}}(\operatorname{Dec}, \operatorname{Enc}_{\mathsf{pk}}(c), \operatorname{Enc}_{\mathsf{pk}}(\mathsf{sk})) \\
    &= \operatorname{Enc}_{\mathsf{pk}}(\operatorname{Dec}_{\mathsf{sk}}(c)) \\
    &= \operatorname{Enc}_{\mathsf{pk}}(m)
\end{align*}
In other words,
$$\operatorname{Dec}_{\mathsf{sk}}(c') = \operatorname{Dec}_{\mathsf{sk}}(\operatorname{Enc}_{\mathsf{pk}}(m)) = m = \operatorname{Dec}_{\mathsf{sk}}(\operatorname{Enc}_{\mathsf{pk}}(m)) = \operatorname{Dec}_{\mathsf{sk}}(c).$$
This means that $c'$ and $c$ are decrypted to the same message, but since many homomorphic computations have been performed on $c$, while $c'$ is new, $c'$ contains less noise. Note, however, that one operation has already been performed, namely the homomorphic decryption (last line of Algorithm \ref{alg:bootstrap}), so it does not contain as little noise as if $m$ was encrypted completely from scratch.

One last note is that bootstrapping, unfortunately, is quite computationally heavy and requires much memory. Therefore, there is huge interest in today's FHE research to increase the efficiency of bootstrapping.


%[TODO: draw image??]

%Read:
%\begin{itemize}
%    \item bs in Gentry's original paper
%    \item AP bs \url{https://eprint.iacr.org/2014/094.pdf}
%    \item GINX bs \url{https://eprint.iacr.org/2014/283.pdf}
%    \item improvements (best of both) \url{https://eprint.iacr.org/2022/198.pdf}
%    \item bootstrapping for FHEW etc \url{https://eprint.iacr.org/2020/086.pdf}
%\end{itemize}


\section{Bootstrapping}\label{sec:bootstrapping}
As seen above, bootstrapping refreshes FHE ciphertexts by evaluating their decryption algorithm homomorphically. In Appendix \ref{sec:fhe_schemes}, we also see that most of the existing FHE schemes decrypt by computing a function $$g(\langle \mathbf{c}, \mathbf{s} \rangle  \mod q) = g(\sum_{i=1}^nc_is_i \mod q ),$$ where $\mathbf{c} \in \mathbb{Z}_q^n$ is the ciphertext, $\mathbf{s} \in \mathbb{Z}_q^n$ is the private key, and $g$ is a function that in some way uses the result of the scalar product to decrypt the ciphertext.

The naive approach of doing bootstrapping is to simply compute $g$ homomorphically as in Algorithm \ref{alg:bootstrap}. However, the modulo operation is computationally expensive \cite{cite:fast_bootstrap_crypto23}. Therefore, in practice, one needs to modify the bootstrapping so that the execution time is reduced. There are several ways of doing so, and one is based on so-called blind rotation. This section will give an overview of how this way of doing bootstrapping usually works. In the next chapter, we introduce a new, faster blind rotation algorithm developed by Xiang et al. \cite{cite:fast_bootstrap_crypto23}.

The main idea of doing bootstrapping with blind rotation is to transform the noisy ciphertext to a ciphertext in another scheme, where the modulo operations can be computed cheaper. More specifically, one wants to transform the ciphertext to a ring-based encryption system, where the ring $\mathcal{R} = \mathbb{Z}[X]/(X^N+1)$ is constructed so that $q=2N$. In this way, the order of $X \in \mathcal{R}$ is $\operatorname{ord}_{\mathcal{R}} (X) =  q$, which means that that the computations modulo $q$ can be computed easily in the exponent:
$$X^{ \sum_i c_i s_i  } =
X ^ {(\sum_i c_i s_i \operatorname{mod} q) + kq} =
X^{\sum_i c_i s_i \operatorname{mod} q} \cdot (X^q)^k =
X^{\sum_i c_i s_i \operatorname{mod} q}.$$
%To read the exponent $X^{ \sum_i c_i s_i  }$, i.e. to get the result of the modulo computation, one multiplies the rotation polynomial $r(X) = \sum_{i=0}^{q-1} iX^{-i}$, and read the coefficient of the constant term:
%$$ \sum_i c_i s_i \equiv \operatorname{coeff}_0 \left(r(X) \cdot X^{ \sum_i c_i s_i  } \right)  \mod q $$
Then, it is just to read the exponent of $X^{ \sum_i c_i s_i  }$ to get the result of the modulo computation. 

Usually, bootstrapping using blind rotation consists of three steps:
\begin{enumerate}[font=\bfseries]
    \item \textbf{Blind rotation} \\ Transforms the first-layer ciphertext to a ring-based ciphertext. 
    \item \textbf{Extraction} \\ Transforms the ring-based ciphertext back to the first-layer encryption again.
    \item \textbf{Key switching} \\ Switches the new first-layer key back to the original key.
\end{enumerate}

There are two common bootstrapping algorithms that are usually used, and they differ mainly in how the blind rotation is performed \cite{cite:hal_bootstrap_intro}. One performs blind rotation using the AP algorithm \cite{cite:ap}, while the other uses the GINX algorithm \cite{cite:ginx}.

Section \ref{sec:ap} and \ref{sec:ginx} below give a short introduction to these algorithms. We refer to \cite{cite:ap} and \cite{cite:ginx} for more details about how AP and GINX work.

Usually, AP needs a larger evaluation key than GINX. GINX is faster for binary and ternary secrets, but slower for larger secrets \cite{cite:hal_bootstrap_intro}.

In Chapter \ref{sec:xzddf_intro}, we present a new algorithm from Xiang et al. \cite{cite:fast_bootstrap_crypto23}, which both has a small evaluation key and is fast for large secrets. Another algorithm, that also beats AP and GINX in memory and time complexity, is called LMKCDEY \cite{cite:lmkcdey_le_et_al_better_ap_ginx}.

\subsection{AP Blind Rotation}\label{sec:ap}
The AP method for doing blind rotation relies on decompositions of $a_i=\sum_j a_{i,j}B^j$ in a base $B$. All possible values of $a_{i,j}s_i \in \mathbb{Z}_q$, where $\mathbf{s}=(s_0, \dots, s_{n-1}) \in \mathbb{Z}_q^n$ is the private key, are pre-computed and encrypted, and then stored in the blind rotation evaluation key $BRK^{\text{AP}}$.

Algorithm \ref{alg:ap_brkgen} shows how the evaluation key is generated, while Algorithm \ref{alg:ap} shows how the blind rotation is performed, where the operator $\otimes$ is the external product between RLWE ciphertexts and RGSW ciphertexts, defined by for example Alperin-Sheriff and Peikert \cite{cite:ap}. The output $\mathsf{ACC}$ is essentially an RLWE ciphertext of the decryption of the inputted LWE ciphertext $c = (\mathbf{a}, b)$.

\renewcommand{\Comment}[2][.35\linewidth]{\leavevmode\hfill\makebox[#1][l]{//~#2}}  %% https://tex.stackexchange.com/questions/180212/how-to-align-comments-in-algorithm-code
\begin{algorithm}[ht]
\caption{\;\;$\mathsf{AP.BRKGen}$}\label{alg:ap_brkgen}
\begin{algorithmic}
\Require
  \State $\mathbf{s} \in \mathbb{Z}_q^n$ \Comment{secret key}
  \State $B \in \mathbb{Z}_q$ \Comment{basis}  
\Ensure $BRK^{\text{AP}}$ \Comment{evaluation key}  
  \For{$i=0 \;..\; (n-1)$}
    \For{$j=0 \;..\; (\log_B(q) - 1)$}
      \For{$v=0 \;..\; B$}
        \State $BRK_{i,j,v}^{\text{AP}} \gets \operatorname{RGSW}_{\mathbf{z}}(X^{vB^js_i})$   \Comment{RGSW encryption}
      \EndFor
    \EndFor  
  \EndFor
\end{algorithmic}
\end{algorithm}

\renewcommand{\Comment}[2][.4\linewidth]{\leavevmode\hfill\makebox[#1][l]{//~#2}}  %% https://tex.stackexchange.com/questions/180212/how-to-align-comments-in-algorithm-code
\begin{algorithm}[ht]
\caption{\;\;$\mathsf{AP.BREval}$}\label{alg:ap}
\begin{algorithmic}
\Require
  \State $f \in \mathcal{R}_Q$
  \State $c = (\mathbf{a}, b) \in \mathbb{Z}_q^n \times \mathbb{Z}_q$ \Comment{LWE ciphertext}
  \State $BRK^{\text{AP}}$
  \State $B \in \mathbb{Z}_q$ \Comment{basis}  
\Ensure $\mathsf{ACC}$ \Comment{accumulator}  
  \State $Y \gets X^{\frac{2N}{q}} \in \mathcal{R}_Q$ \Comment{$\implies \operatorname{ord}_{\mathcal{R}_Q}(Y) = q$}
  \State $\mathsf{ACC} \gets (0, Y^{-b} \cdot f) \in \mathcal{R}_Q \times \mathcal{R}_Q$
  \For{$i=0 \;..\; (n-1)$}
    \For{$j=0 \;..\; (\log_B(q) - 1)$}
      \State $a_{i,j} \gets \lfloor a_i / B^j \rfloor \mod B$
      \State $\mathsf{ACC} \gets \mathsf{ACC} \otimes BRK_{i,j,a_{i,j}}^{\text{AP}} $ \Comment{$\otimes$ is the external product}
    \EndFor  
  \EndFor
\end{algorithmic}
\end{algorithm}




\subsection{GINX Blind Rotation}\label{sec:ginx}
In the GINX method for doing blind rotation, the elements of the private key $\mathbf{s}=(s_0, \dots, s_{n-1}) \in \mathbb{Z}_q^n$ are decomposed as $s_i = \sum_{j=0}^{|U|-1} u_j \cdot s_{i,j}$ where $s_{i,j} \in \{0,1\}$ and $U = \{u_0, \dots, u_{|U|-1}\}$ is a public set. All $s_{i,j}$ are then encrypted and stored in the blind rotation evaluation key $BRK^{\text{GINX}}$.

Algorithm \ref{alg:ginx_brkgen} shows how the evaluation key is generated, while Algorithm \ref{alg:ginx} shows how the blind rotation is performed. Just as for the AP blind rotation, the operator $\otimes$ is the external product between RLWE ciphertexts and RGSW ciphertexts.
%The output $\mathsf{ACC}$ is essentially an RLWE ciphertext of the decryption of the inputted LWE ciphertext $c = (\mathbf{a}, b)$.


% from \cite{lmkcdey....}
\renewcommand{\Comment}[2][.35\linewidth]{\leavevmode\hfill\makebox[#1][l]{//~#2}}  %% https://tex.stackexchange.com/questions/180212/how-to-align-comments-in-algorithm-code
\begin{algorithm}[ht]
\caption{\;\;$\mathsf{GINX.BRKGen}$}\label{alg:ginx_brkgen}
\begin{algorithmic}
\Require
  \State $\mathbf{s} \in \mathbb{Z}_q^n$ \Comment{secret key}
  \State $U \subset \mathbb{Z}_q$ %\Comment{Chosen s.t. $\exists s_j \subset U: $}
\Ensure $BRK^{\text{GINX}}$ \Comment{evaluation key}  
  \For{$i=0 \;..\; (n-1)$}
      \State $s_{i,j} \gets \operatorname{Solve}\left(s_i = \sum_{j=0}^{|U|-1} u_j \cdot s_{i,j} \text{ \;\; s.t. } s_{i,j} \in \{0,1\}\right)$
    \For{$j=0 \;..\; (|U|-1)$}
      \State $BRK_{i,j}^{\text{GINX}} \gets \operatorname{RGSW}_{\mathbf{z}}(s_{i,j})$   \Comment{RGSW encryption}
      
    \EndFor  
  \EndFor
\end{algorithmic}
\end{algorithm}


% from \cite{lmkcdey....}
\renewcommand{\Comment}[2][.4\linewidth]{\leavevmode\hfill\makebox[#1][l]{//~#2}}  %% https://tex.stackexchange.com/questions/180212/how-to-align-comments-in-algorithm-code
\begin{algorithm}[ht]
\caption{\;\;$\mathsf{GINX.BREval}$}\label{alg:ginx}
\begin{algorithmic}
\Require
  \State $f \in \mathcal{R}_Q$
  \State $c = (\mathbf{a}, b) \in \mathbb{Z}_q^n \times \mathbb{Z}_q$ \Comment{LWE ciphertext}
  \State $BRK^{\text{GINX}}$
\Ensure $\mathsf{ACC}$ \Comment{accumulator}  
  \State $Y \gets X^{\frac{2N}{q}} \in \mathcal{R}_Q$ \Comment{$\implies~ \operatorname{ord}_{\mathcal{R}_Q}(Y) = q$}
  \State $\mathsf{ACC} \gets (0, Y^{-b} \cdot f) \in \mathcal{R}_Q \times \mathcal{R}_Q$
  \For{$i=0 \;..\; (n-1)$}
    \For{$j=0 \;..\; (|U| - 1)$}
      \State $\mathsf{ACC} \gets \mathsf{ACC} + (Y^{a_iu_j}-1) \cdot (\mathsf{ACC} \otimes BRK_{i,j}^{\text{GINX}}) $ %\Comment{$\otimes$ is the external product}
    \EndFor  
  \EndFor
\end{algorithmic}
\end{algorithm}



% OLD
% We will now describe each step in detail.

% \section{Blind Rotation}

% AP: large key
% GINX: faster for binary/ternary secrets \cite{cite:hal_bootstrap_intro}, slower for larger secrets. Smaller key.

% \subsection{AP Blind Rotation}

% The AP method for doing blind rotations relies on decompositions of $a_i=\sum_j a_{ij}B^j$ in a base $B$. All possible values of $a_{ij}s_i \in \mathbb{Z}_q$ are precomputed and encrypted, stored in the evaluation key. 

% \subsection{GINX Blind Rotation}

% In the GINX method for doing blind rotation, the elements of the private key are decomposed as $s_i = \sum_{u \in U}s_{i,u}u$ for some public set $U$, where $s_{i,u} \in \{0,1\}$. Then, all $s_{i,u}$ are encrypted and stored in the evaluation key. 

% \subsection{Lee et al. \cite{cite:le_et_al_better_ap_ginx}}

% Rely on automorphisms.

% \section{Key Switching}

% \section{Extraction}


\section{Security}

As can be seen in Algorithm \ref{alg:bootstrap}, an encryption of the private key $\overline{\mathsf{sk}} = \operatorname{Enc}_{\mathsf{pk}}(\mathsf{sk})$ is required when doing bootstrapping. All existing FHE schemes today require this in one way or another \cite{cite:QianFHE}. Therefore, one assumption for FHE schemes to be secure is that it is safe to decrypt a private key under its public key. This assumption is called the \textit{circular security assumption}.

For additional information about the security of FHE schemes, beyond what is necessary for understanding this thesis, we refer to Chapter VI in \cite{cite:QianFHE}.

%\cite{cite:QianFHE}: IND-CPA and CIRCULAR SECURE (encrypting key), function privacy, circuit privacy...
%\cite{cite:QianFHE}: security chapter...
