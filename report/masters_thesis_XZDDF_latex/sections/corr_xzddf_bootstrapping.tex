%As far as the author of this thesis can understand, there is a problem with the XZDDF blind rotation from Xiang et al. \cite{cite:fast_bootstrap_crypto23}, described in Chapter \ref{sec:xzddf_intro} above. In this chapter, we will explain what goes wrong, and then propose a solution to the problem when working with boolean operations.

There is a problem with the XZDDF blind rotation from Xiang et al. \cite{cite:fast_bootstrap_crypto23}. In this chapter, we will first explain what goes wrong in the algorithm, and then propose a solution to the problem when working with Boolean operations.


\section{The Problem}

The problem with the blind rotation from Xiang et al. \cite{cite:fast_bootstrap_crypto23} is related to the rotation polynomial $r(X^{\frac{2N}{q}}) = \sum_{i=0}^{q-1} iX^{- \frac{2N}{q} \cdot i}$. Since we are working in the ring \linebreak $\mathcal{R}_Q := \mathbb{Z}_Q[X]/(X^N+1)$, we have that $X^N \equiv -1$ and $X^{2N} \equiv 1$, so
\begin{equation*}
  X^{-i} =
    \begin{cases}
      1, & \text{if $i=0$} \\
      -X^{N-i}, & \text{if $1 \leq i \leq N$}\\
      X^{2N-i}, & \text{if $N+1 \leq i \leq 2N-1$}.
    \end{cases}       
\end{equation*}
To make things simpler, let us assume that $q=2N$. Then
\begin{align*}
    r(X) &= \sum_{i=0}^{q-1} iX^{-i} \\
    &= -1 \cdot X^{N-1} -2 \cdot X^{N-2} - \dots - N + \\
    & \;\;\;\; + (N+1) \cdot X^{N-1} + (N+2) \cdot X^{N-2} + \dots + (2N-1) \cdot X \\
    & = -N + N \cdot X + N \cdot X^2 + \dots + N \cdot X^{N-1}.
\end{align*}
This means, that when computing the product $r(X) \cdot X^{\operatorname{noised}(m)}$, in general, the coefficient term
$$\operatorname{coeff}_0 \left(r(X) \cdot X^{\operatorname{noised}(m)}\right) \neq \operatorname{noised}(m).$$

The problem is that the second half of the terms in the sum $r(X) = \sum_{i=0}^{q-1} iX^{-i}$ wraps around, and is added to the first half of the terms. This means, that the problem arises not only for $q=2N$ that we assumed, but also for any $q|2N$ and $r(X^{\frac{2N}{q}})$.

An obvious way to avoid the problem with the wrap-around is to just skip the second half of the terms, i.e. defining the rotation polynomial as
$$r(X^{\frac{2N}{q}}) := \sum_{i=0}^{q/2-1} iX^{- \frac{2N}{q} \cdot i}.$$
However, then for messages $\operatorname{noised}(m) \geq \frac{q}{2}$, we still have that
$$\operatorname{coeff}_0 \left(r(X^{\frac{2N}{q}}) \cdot X^{\frac{2N}{q} \cdot \operatorname{noised}(m)}\right) \neq \operatorname{noised}(m).$$
We will now explain how this rotation polynomial can be slightly modified so that we can retrieve all binary plaintext messages.


\section{A Suggestion of Solution}

In this section, a proposal for a solution to the problem above will be suggested. The solution is, however, constrained to the case when working with Boolean operations and binary messages.

Assume that the message is $m \in \mathbb{Z}_2$, and that we are using a Regev-like encryption
$$c = \operatorname{LWE}_{q,\mathbf{s}}(m) = \left(\mathbf{a}, b = \langle \mathbf{a}, \mathbf{s} \rangle + m \cdot \frac{q}{t} + e\right),$$
where $t=4$ (this choice will be explained below).

Let $\diamond$ denote a binary operation that we want to compute on two ciphertexts $c_0 = (\mathbf{a}_0,b_0)$ and $c_1 = (\mathbf{a}_1,b_1)$, i.e. we want to compute $c_0 \diamond c_1$. For example, $\diamond$ can be $\operatorname{OR}$ or $\operatorname{AND}$.

The algorithm then starts by computing
$$c = c_0 + c_1 = (\mathbf{a}_0 + \mathbf{a}_1, b_0 + b_1) =: (\mathbf{a},b).$$
Since $\operatorname{LWE}$ is homomorphic, we now have that
\begin{equation*}
  \operatorname{Dec}(c) =
    \begin{cases}
      0, & \text{if $(m_0, m_1) = (0,0)$} \\
      1, & \text{if $(m_0, m_1) = (0,1)$ or $(1,0)$}\\
      2, & \text{if $(m_0, m_1) = (1,1)$}.
    \end{cases}
\end{equation*}
This is why we set $t=4$ -- we want to be able to handle twos and threes so that we keep the information about whether the sum was an addition of two zeros or two ones.

Now, we will do a modified version of XZDDF bootstrapping on $c$.

Let us denote $\operatorname{noised}(m) := b - \langle \mathbf{a}, \mathbf{s} \rangle = m \cdot \frac{q}{t} + e$, and define $t$ intervals
$$I_i := \left[ i \cdot \frac{q}{t} - \frac{q}{2t},  i \cdot \frac{q}{t} + \frac{q}{2t}\right) \subset \mathbb{Z}_q,$$
for $i \in \{0,1, \dots, t-1 \}$. In our case, we have the $t=4$ intervals
\begin{align*}
    I_0 &= \left[-\frac{q}{8}=\frac{7q}{8}, \frac{q}{8}\right), \\
    I_1 &= \left[\frac{q}{8}, \frac{3q}{8}\right), \\
    I_2 &= \left[\frac{3q}{8}, \frac{5q}{8}\right), \\
    I_3 &= \left[\frac{5q}{8}, \frac{7q}{8}\right).
\end{align*}

If for example $\diamond = \operatorname{OR}$, we now want a function $f_{\operatorname{OR}}$ that maps

\begin{equation*}
  f_{\operatorname{OR}}: \;\; \left (X^{\frac{2N}{q}}\right)^{\operatorname{noised}(m)} \mapsto
    \begin{cases}
      0, & \text{if $\operatorname{noised}(m) \in I_0$} \\
      1, & \text{if $\operatorname{noised}(m) \in I_1$} \\
      1, & \text{if $\operatorname{noised}(m) \in I_2$} \\
      0, & \text{if $\operatorname{noised}(m) \in I_3$}.
    \end{cases}
\end{equation*}

Similarly, if $\diamond = \operatorname{AND}$, we want a function $f_{\operatorname{AND}}$ that maps

\begin{equation*}
  f_{\operatorname{AND}}: \;\; \left (X^{\frac{2N}{q}}\right)^{\operatorname{noised}(m)} \mapsto
    \begin{cases}
      0, & \text{if $\operatorname{noised}(m) \in I_0$} \\
      0, & \text{if $\operatorname{noised}(m) \in I_1$} \\
      1, & \text{if $\operatorname{noised}(m) \in I_2$} \\
      1, & \text{if $\operatorname{noised}(m) \in I_3$}.
    \end{cases}
\end{equation*}

Working cyclically, modulo $q$, the interval $I_0$ can be seen as the following interval to $I_3$. We then see that for both operators $\operatorname{OR}$ and $\operatorname{AND}$, the intervals that are mapped to 0 and 1 consist of two following intervals
\begin{align*}
&I^0 = I_{k} \cup I_{(k+1 \text{ mod 4})}\\
&I^1 = I_{(k+2 \text{ mod 4})} \cup I_{(k+3 \text{ mod 4})},    
\end{align*}
 where $k=3$ for $\operatorname{OR}$ and $k=0$ for $\operatorname{AND}$. In fact, as can be seen in the \texttt{binfhe/} directory of \texttt{OpenFHE} \cite{cite:openFHE}, one can for any binary operation find two intervals $I^0 = [q_0, q_1)$ and $I^1 = [q_1, q_0)$, where $q_1=q_0 + q/2$ (sometimes the binary operation needs to be computed as a composition of other binary operations).

Working in $R_Q$, we know that $X^i$ wraps around negacyclically at $i=N$, so that $aX^i = -aX^{i+N} \mod (X^N+1)$. This property was the cause of failure when using the original rotation polynomial, but it can also be used to solve the problem by forming a new rotation polynomial
        \begin{align}
            r(X^{\frac{2N}{q}}) :&= -\sum_{i=0}^{q/4-1} \left(X^{-\frac{2N}{q}}\right)^i + \sum_{i=q/4}^{q/2-1} \left(X^{-\frac{2N}{q}}\right)^i \notag \\ %\label{eq:new_rot} \\
            &= -1 \cdot \left(X^{-\frac{2N}{q}}\right)^0 - 1 \cdot \left(X^{-\frac{2N}{q}}\right)^1 - \dots - 1 \cdot \left(X^{-\frac{2N}{q}}\right)^{\frac{q}{4}-1} + \notag \\
            & \;\;\;\; + 1 \cdot \left(X^{-\frac{2N}{q}}\right)^{\frac{q}{4}} + 1 \cdot \left(X^{-\frac{2N}{q}}\right)^{\frac{q}{4}+1} + \dots + 1 \cdot \left(X^{-\frac{2N}{q}}\right)^{\frac{q}{2}-1} \notag
        \end{align}
% \begin{align*}
% r(X^{\frac{2N}{q}}) = -1 & \cdot \left(X^{-\frac{2N}{q}}\right)^0 - 1 \cdot \left(X^{-\frac{2N}{q}}\right)^1 - \dots - 1 \cdot \left(X^{-\frac{2N}{q}}\right)^{\frac{q}{4}-1} + \\
% &+ 1 \cdot \left(X^{-\frac{2N}{q}}\right)^{\frac{q}{4}} + \dots + 1 \cdot \left(X^{-\frac{2N}{q}}\right)^{\frac{q}{2}-1}.
% \end{align*}

In this way,
\begin{equation*}
    m':=\operatorname{coeff}_0\left(r(X^{\frac{2N}{q}}) \cdot \left(X^{\frac{2N}{q}(\operatorname{noised}(m)+(\frac{q}{4}-q_1))}\right)\right) = 
    \begin{cases}
      -1, & \text{if $\operatorname{noised}(m) \in [q_0, q_1)$} \\
      1, & \text{if $\operatorname{noised}(m) \in [q_1, q_0)$}.
      % 1, & \text{if $\operatorname{noised}(m) \in [q_1, q_0)$} \\
      % -1, & \text{if $\operatorname{noised}(m) \in [q_0, q_1)$}.
    \end{cases}
\end{equation*}

We compute this constant term just as in the original XZDDF bootstrapping. The final problem to overcome with this algorithm is to find a way to map the constant term back to binary values:
\begin{equation*}
    m' \mapsto
    \begin{cases}
      0, & \text{if $m' = -1$} \\
      1, & \text{if $m' = 1$}.
    \end{cases}
\end{equation*}

This is easily achieved by first choosing $\Delta = \frac{Q}{4} \cdot \frac{1}{2} = \frac{Q}{8}$ in the NTRU encryption. The output from the XZDDF blind rotation is an  $\operatorname{LWE}$ ciphertext on the form
$$c' = \operatorname{LWE}_{Q,\mathbf{f}}(m') = (\mathbf{a}, b' = \langle \mathbf{a}, \mathbf{s} \rangle + \Delta \cdot m' + e)$$
in modulo $Q$ and under the secret key $\mathbf{f}$. We know that $m' \in \{-1,1\}$, and with $\Delta = \frac{Q}{8}$, we get
\begin{equation*}
    c' = \operatorname{LWE}_{Q,\mathbf{f}}(m') =
    \begin{cases}
      (\mathbf{a}, b' = \langle \mathbf{a}, \mathbf{s} \rangle - \frac{Q}{8} + e), & \text{if $m' = -1$} \\
      (\mathbf{a}, b' = \langle \mathbf{a}, \mathbf{s} \rangle + \frac{Q}{8} + e), & \text{if $m' = 1$}. \\
    \end{cases}
\end{equation*}
Finally, we compute a ciphertext
$$c = (\mathbf{a},b) = \left(\mathbf{a},b' + \frac{Q}{8}\right).$$
In this way,
\begin{equation*}
    c = \operatorname{LWE}_{Q,\mathbf{f}}(m') =
    \begin{cases}
      (\mathbf{a}, b = \langle \mathbf{a}, \mathbf{s} \rangle + e), & \text{if $m' = -1$} \\
      (\mathbf{a}, b = \langle \mathbf{a}, \mathbf{s} \rangle + \frac{Q}{4} + e), & \text{if $m' = 1$},
    \end{cases}
\end{equation*}
so that
\begin{equation*}
    m = \operatorname{Dec}(c) =
    \begin{cases}
      0, & \text{if $m' = -1$} \\
      1, & \text{if $m' = 1$},
    \end{cases}
\end{equation*}
which is just the map we wanted. Then, it is just to do the modulo switch and the key switch back to $q$ and $\mathbf{s}$ as usual. Now, the problem has been solved.

\subsection{Summary of Solution}
In summary, we have found a modification of the XZDDF algorithm that works when doing Boolean operations with binary messages, and when the first-layer encryption is Regev LWE encryption. The modified XZDDF bootstrapping is performed by the following steps.
\begin{enumerate}
    \item Set $\Delta = Q/8$.
    \item Add $(q/4-q_1)$ to the last element $b_{\text{{in}}}$ of the inputted LWE ciphertext \linebreak $c_{\text{{in}}}=(\mathbf{a}_{\text{{in}}},b_{\text{{in}}})$, where $q_1$ is depends on the binary operation and can be computed by for example \texttt{OpenFHE}.
    $$c'_{\text{{in}}} = \left(\mathbf{a}_{\text{{in}}},b_{\text{{in}}} + q/4-q_1\right).$$
    \item Use the rotation polynomial $$r(X^{\frac{2N}{q}}) = -\sum_{i=0}^{q/4-1} \left(X^{-\frac{2N}{q}}\right)^i + \sum_{i=q/4}^{q/2-1} \left(X^{-\frac{2N}{q}}\right)^i.$$
    \item Perform XZDDF bootstrapping as usual and extract the LWE ciphertext.
    \item Add $Q/8$ to the last element $b'_{\text{out}}$ of the outputted LWE ciphertext \linebreak $c'_{\text{out}}=(\mathbf{a}_{\text{out}},b'_{\text{out}})$:
    $$c_{\text{out}} = \left(\mathbf{a}_{\text{out}},b'_{\text{out}} + \frac{Q}{8}\right).$$
\end{enumerate}