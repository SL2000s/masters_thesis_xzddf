%%%%%%%% Basic Notations %%%%%%%%
\section{Basic Notations}

In this paper, the set of natural numbers $\mathbb{N}$ is defined as all non-negative numbers (including 0), while $\mathbb{N}^* := \mathbb{N}\backslash\{0\}$.

$\mathcal{R} := \mathbb{Z}[X] / (X^N + 1)$ is a quotient ring of a polynomial ring $\mathbb{Z}[X]$ over the integers, modulo $(X^N + 1)$. If nothing else is specified, $N = 2^d$ for a $d \in \mathbb{N}^*$.

$\mathcal{R}_Q := \mathcal{R} /  Q\mathcal{R} = \mathbb{Z}_Q[X] / (X^N + 1)$, where $\mathbb{Z}_Q := \mathbb{Z} /  Q \mathbb{Z}$. To specify the degree $N$ of the polynomial, $\mathcal{R}_{Q, N}$ is used.
%For $a \in R$, $[a]_q$ refers to $a \mod q$ with coefficients in $(-q/2, q/2]$

All vectors are bold, while elements in polynomial rings are not. The scalar product of two vectors $\mathbf{u}$ and $\mathbf{v}$ is denoted as $\langle \mathbf{u}, \mathbf{v} \rangle$.

$\lfloor x \rfloor$, $\lceil x \rceil$ and $\lfloor x \rceil = \lfloor x + 1/2 \rfloor$ denote the floor function, the ceiling function, and the rounding function, respectively.
%$\lfloor x \rfloor$ denotes the floor function, which outputs the greatest integer less than or equal to $x$, and $\lceil x \rceil$ denotes the ceiling function, i.e. the smallest integer greater than or equal to $x$. The rounding function is denoted as $\lfloor x \rceil := \lfloor x + 1/2 \rfloor$.

$X \xleftarrow{\text{s}} \chi$ means that a random variable $X$ is sampled from a distribution $\chi$. For example, $X \xleftarrow{\text{s}} \mathcal{U}(S)$ denotes a sample $X$ that is uniformly drawn from a set $S$.

%$\norm{r}$ of an element $r \in R$ is the euclidean norm of the coefficients



%%%%%%%% Probability %%%%%%%%
\section{Probability Theory}

Marcolla et al. \cite{cite:QianFHE} define a negligible probability function in the following way.
\begin{definition}[Negligible probability function]
    A probability function \linebreak $\operatorname{negl}(x): \mathbb{Z} \to \mathbb{R}$ is called negligible if, for any $c \in \mathbb{Z}$, there exists an $N \in \mathbb{Z}$, such that $|\operatorname{negl}(x)| < 1/x^c$ for all $x > N$. 
\end{definition}

The opposite of a negligible probability function is an overwhelming probability function.
\begin{definition}[Overwhelming probability function]
    A probability function \linebreak $\operatorname{overwhelm(x)}: \mathbb{Z} \to \mathbb{R}$ is called an overwhelming probability function if and only if $1-\operatorname{overwhelm}(x)$ is a negligible probability function. 
\end{definition}

Now, $B$-boundedness can be defined similarly to the definition of Gentry et al. \cite{cite:gsw}.
\begin{definition}[$B$-boundedness]
    A distribution $\chi$, supported over integers, is called $B$-bounded if
    $$\mathbb{P}_{X \xleftarrow{\text{s}} \chi}[|X| > B] = \operatorname{negl}(x),$$
    where $\operatorname{negl}(x)$ is a negligible function.
\end{definition}
In other words, a distribution $\chi$ is $B$-bounded if the probability of sampling a value $X \xleftarrow{\text{s}} \chi$ greater than $B$ is negligibly small.

Next, the discrete Gaussian distribution \cite{cite:discrete_gaussian} is defined.
\begin{definition}[Discrete Gaussian distribution]
    A discrete distribution $\mathcal{N}_{\mathbb{Z}}(\mu, \sigma^2)$, where $\mu \in \mathbb{Z}$ and $\sigma \in \mathbb{R}$, is called a discrete Gaussian distribution with center $\mu$ and scale $\sigma$, if it has the discrete probability function
    $$\mathbb{P}_{X \xleftarrow{\text{s}} \mathcal{N}_{\mathbb{Z}}(\mu, \sigma^2)}[X=x] = \frac{e^{-(x-\mu)^2/(2\sigma^2)}}{ \sum_{y \in \mathbb{Z}} e^{-(y-\mu)^2/(2\sigma^2)} }.$$
\end{definition}
Note the similarity between the discrete Gaussian distribution $\mathcal{N}_{\mathbb{Z}}(\mu, \sigma^2)$ and the continous Gaussian (normal) distribution $\mathcal{N}(\mu, \sigma^2)$, where the later has the density function
$$ \mathbb{P}_{X \xleftarrow{\text{s}} \mathcal{N}(\mu, \sigma^2)}[X=x] = \frac{1}{\sigma \sqrt{2 \pi}}e^{-\frac{1}{2} (\frac{x-\mu}{\sigma})^2}.$$

%\begin{definition}[$B$-boundedness]
%    An ensamble of distributions $\{\chi_n\}_{n \in \mathbb{N}}$, supported over integers, is called $B$-bounded if
%    $$\mathbb{P}_{e \xleftarrow{\text{s}} \chi_n}[|e| > B] = \operatorname{negl}(n),$$
%    where $\operatorname{negl}(n)$ is a negligible function.
%\end{definition}

%\begin{definition}[$B$-boundedness]
%    A distribution $\chi$ is called $B$-bounded if it is supported on the interval $[-B, B]$.
%\end{definition}



%%%%%%%% Number Theory %%%%%%%%
\section{Number Theory}

To understand where the name fully homomorphic encryption comes from, a homomorphic function is defined.
\begin{definition}[Homomorphic function]
    A function $f: A \to B$ for two algebraic structures $A$ and $B$ of the same type is called homomorphic under the operation $\diamond$ if
    $$f(x \diamond y) = f(x) \diamond f(y).$$
\end{definition}
%\begin{definition}[Homomorphic function]
%    A function $f: A \to B$ for two algebraic structures $A$ and $B$ associated with two operations $\dot$ and $\operatorname{\odot}$, respectively, called homomorphic under operation if
%    $$f(x \diamond y) = f(x) \diamond f(y)$$
%\end{definition}

Moreover, the mathematical structure torus is defined as follows.
\begin{definition}[The real torus]
    The mathematical structure consisting of the real numbers modulo 1 is denoted $\mathbb{T} := \mathbb{R} / \mathbb{Z}$ and is called the real torus.
\end{definition}
This means that the elements of the real torus can be represented by the interval $[0,1)$. Just as the reals, the torus has the addition operation, but the multiplication operation is not defined, and hence, it is not a ring \cite{cite:tfhe_guide}. To see why multiplication is not defined, take for example the element $\left[\frac{1}{3}\right] = \left[\frac{4}{3}\right] \in \mathbb{T}$. Then
$$\left[\frac{1}{3}\right] \cdot \left[\frac{1}{3}\right] = \left[\frac{1}{9}\right] \neq \left[\frac{7}{9}\right] = \left[\frac{16}{9}\right] = \left[\frac{4}{3}\right] \cdot \left[\frac{4}{3}\right].$$
However, multiplication between elements of the torus and integers is well-defined.



%%%%%%%% Lattices %%%%%%%%
\section{Lattices}
Most FHE schemes rely on hard lattice problems. This section contains some theory about lattices.

\begin{definition}[Lattice]
For any two positive integers $k, n \in \mathbb{N}^*$ where $k \leq n$, let $B=(\mathbf{b}_1, ... , \mathbf{b}_k)$ be $k$ linearly independent vectors in $\mathbb{R}^n$. Then the $k$-dimensional lattice $\mathcal{L}$ generated by $B$ is defined as
$$\mathcal{L} = \mathcal{L}(B) = \left\{ \sum_{i=1}^k \gamma_i \mathbf{b}_i : \; \gamma_i \in \mathbb{Z} \right\}.$$
$B$ is called the base of the lattice, and $k$ is called the rank.
\end{definition}

Each lattice $\mathcal{L}$ is associated to a problem called the shortest vector problem, which is defined below.

\begin{definition}[Shortest vector problem]\label{def:svp}
Given a lattice $\mathcal{L}$ and a norm $\norm{\cdot}$ (usually the $L^2$ norm), the problem of finding
$$\argmin_{\mathbf{v} \in \mathcal{L}\backslash\{\mathbf{0}\}} \norm{\mathbf{v}}$$
is called the shortest vector problem $\mathsf{(SVP)}$.
\end{definition}

The shortest vector problem can be generalized to the closest vector problem.

\begin{definition}[Closest vector problem]\label{def:cvp}
Given a lattice $\mathcal{L}$, a norm $\norm{\cdot}$ (usually the $L^2$ norm), and a vector $\mathbf{t} \in \mathbb{R}^n$, the problem of finding
$$\argmin_{\mathbf{v} \in \mathcal{L}} \norm{\mathbf{v}-\mathbf{t}}$$
is called the closest vector problem $\mathsf{(CVP)}$. The expression $\min_{\mathbf{v} \in \mathcal{L}} \norm{\mathbf{v}-\mathbf{t}}$ is called the distance betwen $\mathbf{t}$ and $\mathcal{L}$, and is denoted by  $ \textnormal{dist}(\mathbf{t}, \mathcal{L}) := \min_{\mathbf{v} \in \mathcal{L}} \norm{\mathbf{v}-\mathbf{t}}$.
\end{definition}

There is also a problem called the short integer solution.
\begin{definition}[Short integer solution]\label{def:sis}
Let $q, m, n \in \mathbb{N}^*$ be three positive integers, and take a matrix $A \in \mathbb{Z}_q^{m \times n}$. Moreover, let $\beta \in \mathbb{R}$ be a real number such that $\beta < q$, and let $\norm{\cdot}$ be some norm. Then the problem of finding a non-zero vector $\mathbf{x} \in \mathbb{Z}^n \backslash \{\mathbf{0}\}$ such that
\begin{enumerate}[label=(\roman{*}),labelsep=2em]
\itemindent=140pt
\item $\norm{\mathbf{x}} \leq \beta$ \label{misc:sis_norm_cond}
\item $A \mathbf{x} \equiv \mathbf{0} \mod q,$ \label{misc:sis_eq}
\end{enumerate}
is called the short integer solution problem and is abbreviated $\mathsf{SIS}_{n,m,q,\beta}$.
\end{definition}

% Note that without condition \ref{misc:sis_norm_cond} in Definition \ref{def:sis}, it is easy to find a solution to \ref{misc:sis_eq} by simple Gaussian elimination. Moreover, $\beta < q$ is to avoid a solution like $\mathbf{x} = (q, 0, 0, ...)'$.

%If SVP is hard in the worst case, then SIS is hard on average. [ Miklós Ajtai...]

At last, a problem that partly relies on a version of the $\mathsf{SVP}$ problem is defined. See Marcolla et al. \cite{cite:QianFHE} for more details.
\begin{definition}[Sparse subset sum problem]\label{def:sssp}
Let $S = \{a_1, \dots, a_n\} \subseteq \mathbb{Z}$ be a set of integers. Then the problem of finding a subset $A \subseteq S$ such that
$$\sum_{x \in A} x = 0$$
is called the sparse subset sum problem $(\mathsf{SSSP})$.
\end{definition}



%%%%%%%% Lattice-based Cryptography %%%%%%%%
\section{Lattice-Based Cryptography} %\section{Learning With Errors}

This section gives an introduction to lattice-based cryptography.

\subsection{Hard Problems}

The modern lattice-based cryptosystems are usually based on the learning with errors ($\mathsf{LWE}$) problem. The $\mathsf{LWE}$ problem was introduced by Regev in 2005 \cite{cite:lwe_first}, and is defined below.

\begin{definition}[Learning with errors problem]\label{def:lwe_prob}
For two positive integers \linebreak $q,n \in \mathbb{N}^*$, let $\mathbf{a} \in \mathbb{Z}_q^n$, and $b \in \mathbb{Z}_q$. Then, the problem of finding a vector $\mathbf{s} \in  \mathbb{Z}_q^n$ such that
$$ b = \langle \mathbf{a}, \mathbf{s} \rangle + e \mod q,$$
for some sample $e \xleftarrow{\text{s}} \chi$ drawn from an error distribution $\chi$ over $\mathbb{Z}$, is called the learning with errors problem $(\mathsf{LWE})$.
\end{definition}

There is also a decisional version of the $\mathsf{LWE}$ problem.
\begin{definition}[Decisional learning with errors problem]\label{def:dlwe_prob}
For two positive integers $q,n \in \mathbb{N}^*$, and an error sample $e \xleftarrow{\text{s}} \chi$ drawn from an error distribution $\chi$ over $\mathbb{Z}$, the problem of distinguishing the tuple $(\mathbf{a}, b = \langle \mathbf{a}, \mathbf{s} \rangle + e \mod q) \in \mathbb{Z}_q^n \times \mathbb{Z}_q$ from a tuple uniformly chosen from $\mathbb{Z}_q^n \times \mathbb{Z}_q$ is called the decisional learning with errors problem $(\mathsf{DLWE})$.
\end{definition}

The $\mathsf{LWE}$ and the $\mathsf{DLWE}$ problems are reducible to each other \cite{cite:tfhe}, and the $\mathsf{LWE}$ problem can be reduced to the worst-case hardness of $\mathsf{SVP}$ \cite{cite:bv_lwe}.

The $\mathsf{LWE}$ problem also has a ring-version.

\begin{definition}[Ring learning with errors problem]\label{def:rlwe_prob}
For two positive integers $Q,N \in \mathbb{N}^*$, let $\mathcal{R}_Q := \mathbb{Z}_Q[X] / (X^N + 1)$. Moreover, let $a, b \in \mathcal{R}_Q$. Then, the problem of finding a ring element $s  \in \mathcal{R}_Q$ such that
$$b = a \cdot s + e,$$
for some sample $e \xleftarrow{\text{s}} \chi$ drawn from an error distribution $\chi$ over $\mathcal{R}_Q$, is called the ring learning with errors problem $(\mathsf{RLWE})$.
\end{definition}

\begin{definition}[Decisional ring learning with errors problem]\label{def:drlwe_prob}
For two positive integers $Q,N \in \mathbb{N}^*$, a ring $\mathcal{R}_Q := \mathbb{Z}_Q[X] / (X^N + 1)$, and an error sample $e \xleftarrow{\text{s}} \chi$ drawn from an error distribution $\chi$ over $\mathcal{R}_Q$, the problem of distinguishing the tuple $(a, b = a \cdot s + e) \in \mathcal{R}_Q \times \mathcal{R}_Q$ from a tuple uniformly chosen from $\mathcal{R}_Q \times \mathcal{R}_Q$ is called the decisional ring learning with errors problem $(\mathsf{DRLWE})$.
\end{definition}

At last, we will also define the decisional $\mathsf{NTRU}$ problem.
% \begin{definition}[$\mathsf{NTRU}$ problem]\label{def:ntru_prob}
% For two positive integers $Q,N \in \mathbb{N}^*$, let $\mathcal{R}_Q := \mathbb{Z}_Q[X] / (X^N + 1)$. Then, given $c \in \mathcal{R}_Q$, the problem of finding a ring element $f \in \mathcal{R}_Q$ such that
% $$c = g \backslash f,$$
% where $g \xleftarrow{\text{s}} \chi$  is a sample drawn from an error distribution $\chi$ over $\mathcal{R}_Q$ is called the $\mathsf{NTRU}$ problem.
% \end{definition}

\begin{definition}[Decisional $\mathsf{NTRU}$ problem]\label{def:dntru_prob}
For two positive integers \linebreak $Q,N \in \mathbb{N}^*$, a ring $\mathcal{R}_Q := \mathbb{Z}_Q[X] / (X^N + 1)$, and an error sample $g \xleftarrow{\text{s}} \chi$ drawn from an error distribution $\chi$ over $\mathcal{R}_Q$, the problem of distinguishing $g  /  f \in \mathcal{R}_Q$ from a random polynomial uniformly drawn from $\mathcal{R}_Q$ is called the decisional $\mathsf{NTRU}$ problem.
\end{definition}

%\begin{definition}[Learning with errors problem]\label{def:lwe_prob}
%Given a vector $\mathbf{b} \in \mathbb{Z}_q^m$ and a matrix $A \in \mathbb{Z}_q^{m \times n}$, the problem of finding a vector %$\mathbf{s} \in  \mathbb{Z}_q^n$ such that
%$$A\mathbf{s} + \mathbf{e} = \mathbf{b} \mod q,$$
%where $\mathbf{e} \in \mathbb{Z}_q^m$ is a sample drawn coordinate-wise from an error distribution $\chi$, is called the learning with %errors problem $(\mathsf{LWE})$.
%\end{definition}

% from \cite{cite:gsw}
%\begin{definition}[Decisional $\mathsf{LWE}$ problem]\label{def:dlwe_prob}
%Given a security parameter $\lambda$, let $n=n(\lambda) \in \mathbb{N}^*$ be the dimension, let $q = q(\lambda) > 2$ be an integer, and %let $\chi = \chi(\lambda)$ be a distribution over $\mathbb{Z}$. Consider two distributions, where the first one samples $(\mathbb{a}_i, b_i)$ uniformly random from $\mathbb{Z}_q^{n+1}$ and the second first draws $\mathbb{s} \in \mathbb{Z}_q^n$ uniformly random, and then samples $(\mathbb{a}_i, b_i) \mathbb{Z}_q^{n+1}$ where $\mathbb{a}_i \in \mathbb{Z}_q^n$ is uniformly random, $e_i$ is drawn form $\chi$ and $b_i = \langle \mathbb{a}_i , \mathbb{s}_i \rangle + e_i$. The problem of distinguishing the two distributions given a set of samples is called the decision version of the $\mathsf{LWE}_{n,q,\chi}$ problem, or shorter, the $\mathsf{DLWE}_{n,q,\chi}$ problem.


\subsection{Encryption Algorithms}

$\mathsf{LWE}$-based encryption can be written on the form defined below.
\begin{definition}[LWE ciphertext]\label{def:lwe_encr}
For two positive integers $q,n \in \mathbb{N}^*$, let \linebreak $m \in \mathbb{Z}_q$ be a message and let $\mathbf{a}, \mathbf{s} \in \mathbb{Z}_q^n$ be a public vector and a private key, respectively. Then the LWE encryption of $m$ is defined as
$$\operatorname{LWE}_{q,\mathbf{s}}(m) = (\mathbf{a}, b = \langle \mathbf{a}, \mathbf{s} \rangle + \operatorname{noised}(m)) \in \mathbb{Z}_q^n \times \mathbb{Z}_q,$$
where $\operatorname{noised}(m)$ is a noised encoding of $m$ using some noise $e \xleftarrow{\text{s}} \chi$ drawn from an error distribution $\chi$ over $\mathbb{Z}_q$. %The vector $\mathbf{s}$ is the private key.
\end{definition}

Regev \cite{cite:lwe_first} uses $\operatorname{noised}(m) = m \cdot \frac{q}{t} + e$, so that
\begin{equation}\label{eq:regev_lwe}
    \operatorname{LWE}_{q,\mathbf{s}}^{\text{Regev}}(m) = \left(\mathbf{a}, b = \langle \mathbf{a}, \mathbf{s} \rangle + m \cdot \frac{q}{t} + e\right).
\end{equation}

Note that knowing the private key $\mathbf{s}$, a ciphertext $\operatorname{LWE}_{q,\mathbf{s}}(m) = (\mathbf{a}, b)$ is easily decrypted:
$$ \operatorname{noised}(m) = b - \langle \mathbf{a}, \mathbf{s} \rangle.$$

It is also possible to encrypt using the $\mathsf{RLWE}$ problem.

\begin{definition}[RLWE ciphertext]\label{def:rlwe_encr}
For two positive integers $Q, N \in \mathbb{N}^*$, let $m, a, s \in \mathcal{R}_Q$, be a message, a public value, and a private key, respectively. Then the RLWE encryption of $m$ is defined as
$$\operatorname{RLWE}_{Q,s}(m) = (a, b = a \cdot s + \operatorname{noised}(m)),$$
where $\operatorname{noised}(m)$ is a noised encoding of $m$ using some noise $e \xleftarrow{\text{s}} \chi$ drawn from an error distribution $\chi$ over $\mathcal{R}_Q$.
\end{definition}

Encryption using the $\mathsf{NTRU}$ problem is defined in Chapter \ref{sec:xzddf_intro}.